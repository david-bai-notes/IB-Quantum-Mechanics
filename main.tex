\documentclass[a4paper]{article}

\usepackage{hyperref}

\newcommand{\triposcourse}{Quantum Mechanics}
\newcommand{\triposterm}{Michaelmas 2020}
\newcommand{\triposlecturer}{Dr. M. Ubiali}
\newcommand{\tripospart}{IB}

\usepackage{amsmath}
\usepackage{amssymb}
\usepackage{amsthm}
\usepackage{mathrsfs}

\usepackage{tikz-cd}

\theoremstyle{plain}
\newtheorem{theorem}{Theorem}[section]
\newtheorem{lemma}[theorem]{Lemma}
\newtheorem{proposition}[theorem]{Proposition}
\newtheorem{corollary}[theorem]{Corollary}
\newtheorem{problem}[theorem]{Problem}
\newtheorem*{claim}{Claim}

\newtheorem*{postulate}{Postulate}

\theoremstyle{definition}
\newtheorem{definition}{Definition}[section]
\newtheorem{conjecture}{Conjecture}[section]
\newtheorem{example}{Example}[section]

\theoremstyle{remark}
\newtheorem*{remark}{Remark}
\newtheorem*{note}{Note}

\title{\triposcourse{}
\thanks{Based on the lectures under the same name taught by \triposlecturer{} in \triposterm{}.}}
\author{Zhiyuan Bai}
\date{Compiled on \today}

%\setcounter{section}{-1}

\begin{document}
    \maketitle
    This document serves as a set of revision materials for the Cambridge Mathematical Tripos Part \tripospart{} course \textit{\triposcourse{}} in \triposterm{}.
    However, despite its primary focus, readers should note that it is NOT a verbatim recall of the lectures, since the author might have made further amendments in the content.
    Therefore, there should always be provisions for errors and typos while this material is being used.
    \tableofcontents
    \section{Historical Introduction}
\subsection{Black-body radiation}
The birth of quantum mechanics finds its way through its ability to explain phenomena that cannot be explained in classical physics.
The first example of which is the black-body radiation.
\begin{definition}
    A black body is a totally absorbing heated body at thermal equilibrium $T$.
\end{definition}
What people towards the end of 19th century intended to measure was the energy density of light emitted by a black body at this equilibrium temperature $\rho(\nu,T)$ where $\nu$ is the frequency of the light.
Experimentally, the graph of $\rho$ against $\nu$ fixing $T$ always looks like a curve that grows parabolically at the beginning near $0$ but decreases exponentially after some point.\\
Many physicists tried to model this curve.
Rayleight (1900) made the first attempt in the problem.
He assumed that the black body is a cobe with side length $L$.
The radiation is taken as a superposition of plane waves in the form $e^{i\underline{k}\cdot\underline{x}}$ where $\underline{k}$ is the called the wavenumber of the plane wave.
Obviously, we have $|\underline{k}|=2\pi/\lambda$, so $\nu=|\underline{k}|x/(2\pi)$.\\
However, we cannot take the wavenumbers arbitarily, as we require the wave to vanish at the boundary to make a thermal equilibrium.
With this condition, we obtain
$$\underline{k}=\frac{2\pi}{L}\underline{n},\underline{n}\in\mathbb Z^3\implies \nu=\frac{|\underline{n}|c}{L}$$
Let $N(\nu)$ be the number of modes (possible values of $\underline{n}$) between $\nu$ and $[\nu+\mathrm d\nu]$.
We then obtain
$$N(\nu)\,\mathrm d\nu=2\times 4\pi|\underline{n}|^2\,\mathrm d|\underline{n}|=8\pi\left(\frac{L}{c}\right)^3\nu^2\,\mathrm d\nu$$
We have, let $\langle E(\nu,T)\rangle$ be the average energy of the modes,
$$\rho(\nu,T)=N(\nu)\frac{\langle E(\nu,T)\rangle}{L^3}$$
In classical statistical dynamics, $\langle E\rangle$ is directly proportional to $T$ and does not depend on $\nu$.
So $\langle E\rangle=k_BT$ wherre $k_B$ is the Boltzmann constant.
Where does it come from?
It comes from the fact that the probability for a plane wave to have energy $E$ is
$$\mathbb P(E)\propto e^{-E/(k_BT)}\implies \langle E\rangle=\frac{\int_0^\infty Ee^{-E/(k_BT)}\,\mathrm dE}{\int_0^\infty e^{-E/(k_BT)}\,\mathrm dE}=k_BT$$
However, this shall yield us the energy density
$$\rho(\nu,T)=\frac{8\pi k_B}{c^3}T\nu^2$$
which is obviously NOT compliant with experimental results except for small frequencies.
This is known as the ultraviolet catastrophe, getting its name due to the fact that this model stops working pass the ultraviolt frequencies.\\
Planck (1905) found a workaround of the problem.
He postulated that the energy is discrete instead of continuous, so $E=nh\nu$ for nonnegative integer $n$ and constant $h$ (known as the planck constant).
This is known as a quantisation of the energy of radiation.\\
In this case, we obtain
$$\langle E\rangle=\frac{\sum_{n=0}^\infty nh\nu e^{-nh\nu/(k_BT)}}{\sum_{n=0}^\infty e^{-nh\nu/(k_BT)}}=\frac{h\nu}{e^{h\nu/(k_BT)}-1}$$
which is fascinatingly different from the continuous case.
\footnote{Well, not really.}
In this case, we obtain
$$\rho(\nu,T)=\frac{8\pi h}{c^3}\frac{\nu^3}{e^{h\nu/(k_BT)}-1}$$
which is surprisingly fitting the experimental data.
This makes people start thinking about taking certain physical quantities as discrete objects.
\subsection{Photoelectric Effect}
A long time ago, people have observed the emission of electrons when light is shine on metal surfaces.
In a classical point of view, if the intensity $I$ of the incident light is high enough, the radiation would give enough energy for electrons to emit.
However, this is not what happened in an experiment.
If the frequency is too low, no matter how intense the radiation is, there can be no emission of electron.
Even more puzzling, the rate of the emission of the electrons is not growing with the frequency of the light, but instead with intensity of the light.\\
Einstein (1905) took up the idea of Planck ad came up with the idea of viewing light as light quanta (photons), which are particles.
And each of those particles carries energy $E=h\nu=\hbar\omega$ where $\hbar=h/(2\pi)$ is the reduced Planck constant and $\omega=2\pi\nu$ is the angular frequency.
Photoelectric effect happens, as Einstein interpreted it, when an electron absorbs a photon with large enough energy to make it escape its original structure.\\
So the energy of the emitted electron is bounded by $E\le \hbar\omega-\phi$ where $\phi$ is the ``work function'' that denotes the minimal amount of energy of a single photon spent to trigger photoelectric effect.
With this theory, one can easily explain the observations of photoelectric effect.\\
This theory of viewing light as particles got developed further and helped explaining more phenomena.
In 1923, an experiment known as Compton scattering was conducted by Compton.
In this experiment, Compton used X-Rays to scatter off free electrons.
Classically, if the light is simply a wave, then there will be a re-radiation of light by electrons with a certain angular distribution, which one can obtain as $I(\nu)\propto(1+\cos^2\theta)$.
So one could expect to find a single peak at $\nu_0$ in the graph of $I$ against $\nu$.\\
However, what is observed experimentally, there was a second peak $\nu'$ such that $|\nu_0-\nu'|$ is dependent on $\theta$.\\
In quantum mechanics, if we take lights as particles, as we did in Dynamics \& Relativity, it can be explained.
Assume the deflected electron and photon (whose energy is $E'=\hbar\omega'$) make angles of $\alpha$ $\theta$ to the original path, then by using energy-momentum conservation one can obtain
$$\frac{1}{\omega'}=\frac{1}{\omega}+\frac{\hbar}{mc}(1-\cos\theta)$$
which is consistent with the experimental result.
This also shows that $\hbar$ actually measures the strength of quantum effect.
\subsection{Particles as Waves}
In previous sections, we found that light, which always manifest itself as a wave, can be interpreted as particles too.
A natural thought from that is then:
Can we interpret particles as waves?\\
In 1923, de Broglie postulated that for any particle of mass $m$, it can be associated with a wave with angular frequency $\omega=E/\hbar$ and $\underline{k}=\underline{p}/\hbar$.
People then attempt to conduct experiments to verify this theory.
For example, one can do a diffraction experiment with particle beams.
But experiments of this are usually complicated to administrate as the wavelength of particles can be very small.
But some physicists managed to get it done on small particles like electrons, which does yield a typical diffraction pattern and the calculated wavelength is complicant with de Broglie's theory.
So we can interpret particles as waves.\\
But the electron can't possibly split in space, yet even we pass the electron one-by-one, the same phenomenon happens.
This can be explained using a probabilistic interpretation of the whole thing, which we will cover later.
\subsection{Atomic Models}
In 1897, J. J. Thompson made the first atomic model by viewing atoms as a cloud of positive charges that enclose a number of electrons.
To verify (or disprove) this model, Rutherfold suggested Geiger and Marsden to conduct a scattering experiment.
They took a beam of alpha particles and shine it on a piece of gold foil.
If the model of J. J. Thompson was correct, then the alpha particles should not be deflected by any large angle, but should merely pass through the foil with very small deflection.
However, in the experiment, many deflections with very large angle were observed, and some were even scattered back.\\
After this experiment, Rutherfold came up with another model, which postulates that electrons are orbitting a common nucleus with positive charge.
This explains the phenomenon that was observed in the scattering experiment.
But there are several problems with it:
First, why is there not any radiation from the electrons resulting from the centripetal acceleration?
This radiation should make the electron's energy decrease and eventually collapse, but in reality no such thing happen.
Secondly, why is there discrete instead of continuous frequencies in the atomic spectra?
That is, when an excited atom de-excite, the angular frequencies of light emitted (the line spectra) are discrete and satisfy
$$\omega_{mn}=2\pi cR_0\left( \frac{1}{n^2}-\frac{1}{m^2} \right)$$
for $m>n$.
Here $R_0$ is a constant known as the Rydberg constant, which has order of magnitude $10^7{\rm m^{-1}}$.\\
These questions prompted another reformulation of the atomic model by Bohr (1913).
He hypothesise that the electron orbits are quantised so that the orbital angular momentum $L$ satisfies $L=\hbar n$ for $n\in\mathbb N$.
It sounds out of nowhere, but it works.
There is an explanation of it by taking electrons as waves.
Then, the wavenumber of an electron is just $\underline{k}=\underline{p}/\hbar$ so $\lambda=2\pi\hbar/|\underline{p}|$.
If we imagine the electron in the orbit of radius $r$, then we can take it as a stationary wave on this circumference, so
$$2\pi r=n\lambda=n\frac{2\pi\hbar}{|\underline{p}|}\implies L=|\underline{r}||\underline{p}|=\hbar n$$
which is just Bohr's hypothesis.\\
What is the consequences of this formulation of atomic model?
We have, by Newton's second law,
$$\frac{e^2}{4\pi\epsilon_0}\frac{1}{r^2}=m_e\frac{v^2}{r}$$
in the $\hat{r}$ component.
Hence by $L=\hbar n$, the velocity and radius are quantized as
$$v_n=\frac{n\hbar}{m_e r_n},r_n=n^2\left( \frac{4\pi\epsilon_0}{m_ee^2}\hbar^2 \right)$$
$a_0=r_1$ is called the Bohr radius.
The energy of the electron is also quantised.
$$E_n=\frac{1}{2}m_ev_n^2-\frac{e^2}{4\pi\epsilon_0}\frac{1}{r_n}=-\frac{e^2}{8\pi\epsilon_0a_0}\frac{1}{n^2}=\frac{E_1}{n^2}$$
We have $E_1=-13.6{\rm eV}$ upon calculation.
For an electron with $n>1$, we say it is excited as such states can be created by exciting (giving energy to, usually via photon) the electron in ground state, while an electron with $n=0$ is said to be in ground state.\\
In this model, an de-excitation from state $m$ to $n$ would then give light with angular frequency
$$\omega_{mn}=(E_m-E_n)\frac{1}{\hbar}=2\pi c\frac{m_ec}{2\hbar}\left(\frac{e^2}{4\pi\epsilon_0\hbar c}\right)^2\left( \frac{1}{n^2}-\frac{1}{m^2} \right)$$
which gives a prediction of $R_0$ very close to the experiment.
In particular, calculation reveals that $\omega_{n+1,n}\sim v_n/r_n$ as $n\to\infty$, which can be viewed as a kind of classical limit.
%$$\omega_{n+1,n}=\frac{m_e^3e^4}{64\pi^4\epsilon_0^2\hbar^3}\left( \frac{1}{n^2}-\frac{1}{(n+1)^2} \right)\sim\frac{m_e^3e^4}{32\pi^4\epsilon_0^2\hbar^3}\frac{1}{n^3}=\frac{v_n}{r_n}$$
%as $n\to\infty$.
%This is exactly the classical case.
\footnote{However, $\omega_{n+2,n}$ does not tend to this limit when $n\to\infty$. It does not tend to any well-known classical limit.}
\subsection{Conclusion}
Quantum mechanics is a new framework of explaining physical phenomena.
It may come in counter-intuitive as we always think in the classical way, but it has an undeniable predictivity.
    \section{Foundation of Quantum Mechanics}
\subsection{Quantum Mechanics of a Particle}
In classical dynamics, a particle is characterised by its position $\underline{x}$ and momentum $\underline{p}$.
We formulate classical laws in this framework.
E.g. in Newton's Second Law, we have $F(\underline{x})=m\underline{\ddot{x}}$.
A simple consequence of this is that the position and momentum of a particle at $t=0$ determines its motion under physical laws formulated in this way.\\
But in Quantum Mechanics, things are different.
\begin{postulate}[State and Wavefunction]
    A particle in space is described by its state $\psi:\mathbb R^3\times\mathbb R\to\mathbb C$ such that
    $$\int_{\mathbb R^3}|\psi(\underline{x},t)|^2\,\mathrm dV=N\in\mathbb R_+$$
    The probability amplitude of finding a particle at a place $\underline{x}$ and a given time $t$ is given by the state as a wavefunction $\psi(\underline{x},t)$.
\end{postulate}
There are subtle differences when we talk about state and when we talk about a wavefunction.
But we don't care.
\begin{postulate}[Born's Rule]
    $|\bar\psi(\underline{x},t)|^2\,\mathrm dV$ is the probability of finding the particle in $\mathrm dV$.
    Here, $\bar\psi$ is the normalised wavefunction $\bar\psi=\psi/\sqrt{N}$.
\end{postulate}
\begin{postulate}[Time-Dependent Schr\"odinger's Equation (TDSE)]
    For a particle of mass $m$ in potential $U(\underline{x})$, in the time evolution of $\psi$ we have
    $$i\hbar\frac{\partial\psi}{\partial t}=-\frac{\hbar^2}{2m}\nabla^2\psi+U\psi$$
\end{postulate}
Observe that the equation is first order in $t$ and second order in $\underline{x}$.
It is also linear, so it still holds if we replace $\psi$ by a scalar multiple of it, for example $\bar\psi$.
Heuristically, one can derive this equation (in dimension one) in the following way:\\
Consider the particle as a de Broglie wave with wavefunction $e^{i(kx-\omega t)}$.
So for a free particle with $U(x)=0$, we have $E=p^2/2m$, therefore the wavefunction is
$$\exp(i(kx-\omega t))=\exp\left( \frac{1}{\hbar}(px-Et) \right)=\exp\left( \frac{1}{\hbar}\left(px-\frac{p^2}{2m}t\right) \right)$$
which can be verified a solution to the equation.
But this is just heuristic, the de Broglie wave is not even normalisable.
We will (hopefully) see the actual stuff later.\\
One thing that can easily go wrong with these postulate is that we do not know if the normalised wavefunction will continue to be normalised as time goes, so we need some work on that.
It suffices show that TDSE guarantees that $N$ does not depend on $t$.
\begin{proposition}
    $N$ does not depend on $t$ assuming TDSE.
\end{proposition}
The proof involves the use of some techniques that need some analytical justification.
But this is an applied course, so nobody cares.
\begin{proof}
    Just differentiate
    \begin{align*}
        \frac{\mathrm dN}{\mathrm dt}&=\frac{\mathrm d}{\mathrm dt}\int_{\mathbb R^3}|\psi(\underline{x},t)|^2\,\mathrm dV\\
        &=\int_{\mathbb R^3}\frac{\partial}{\partial t}\left( \psi^*(\underline{x},t)\psi(\underline{x},t) \right)\,\mathrm dV\\
        &=\int_{\mathbb R^3}\psi^*\frac{\partial\psi}{\partial t}+\psi\frac{\partial \psi^*}{\partial t}\,\mathrm dV\\
        &=\int_{\mathbb R^3}\psi^*\left( \frac{i\hbar}{2m}\nabla^2\psi+\frac{i}{\hbar}U\psi\right)+\psi\left( -\frac{i\hbar}{2m}\nabla^2\psi^*-\frac{i}{\hbar}U\psi^* \right)\,\mathrm dV\\
        &=\frac{i\hbar}{2m}\int_{\mathbb R^3}\psi^*\nabla^2\psi-\psi\nabla^2\psi^*\,\mathrm dV\\
        &=\frac{i\hbar}{2m}\int_{\mathbb R^3}\nabla\cdot( \psi^*\nabla\psi-\psi\nabla\psi^*)\,\mathrm dV\\
        &=\frac{i\hbar}{2m}\int_{\partial V}(\psi^*\nabla\psi-\psi\nabla\psi^*)\cdot\mathrm d\underline{S}\\
        &=0
    \end{align*}
    by our boundary condition.
\end{proof}
\begin{remark}
    Assume that $\psi$ is normalised.
    Write $\rho(\underline{x},t)=|\psi(\underline{x},t)|^2$ as the probability density and
    $$\underline{J}=-\frac{i\hbar}{2m}(\psi^*\nabla\psi-\psi\nabla\psi^*)$$
    the probability current.
    Then we have the conservation law
    $$\frac{\partial\rho}{\partial t}+\nabla\cdot\underline{J}=0$$
    by the calculation involved in the above proof.
\end{remark}
\subsection{Principle of Superpositions}
As TDSE is linear in $t$, for states $\phi_1,\phi_2$ satisfying it, $a_1\phi_1+a_2\phi_2$ also satisfies TDSE for any $a_1,a_2\in\mathbb C$.
Obviously, we want to show that this superposition is either $0$ or also normalisable.
\begin{proposition}
    If $\phi_1,\phi_2$ are normalisable, so is $a_1\phi_1+a_2\phi_2$ for any $a_1,a_2\in\mathbb C$ given that it is nonzero.
\end{proposition}
\begin{proof}
    Quite obvious.
\end{proof}
\begin{corollary}
    The set of all states, together with the zero function, is a vector space.
\end{corollary}
This vector space is often denoted by $\mathcal H$.
\begin{proof}
    Immediate.
\end{proof}
We can equip a complex inner product
$$(\psi,\phi)=\int_{\mathbb R^3}\psi^*\phi\,\mathrm dV$$
on this space.
If we also know that it is complete with respect to that inner product, then $\mathcal H$ is a (complex) Hilbert space.
\subsection{Expectation Values and Operators}
For a (one dimensional) particle in state $\psi$, the average value of its position $x$ is then
$$\langle x\rangle=\int_{\mathbb R}x\rho(x,t)\,\mathrm dx=\int_{\mathbb R}x|\bar\psi(x,t)|^2\,\mathrm dx$$
which can be interpreted as the average of repeated measurement on an ensemble of identically prepared systems.
Classically $p=m\dot{x}$ is the momentum.
In quantum mechanics, we define the average momentum to be
$$\langle p\rangle=m\frac{\mathrm d\langle x\rangle}{\mathrm dt}=m\frac{\mathrm d}{\mathrm dt}\int_{\mathbb R}x|\bar\psi(x,t)|^2\,\mathrm dx$$
We can calculate it and using the conservation rule we found earlier and assuming the boundary condition that $\psi$ decays fast enough as $x\to\pm\infty$,
\begin{align*}
    \langle p\rangle&=m\frac{\mathrm d}{\mathrm dt}\int_{\mathbb R}x|\bar\psi(x,t)|^2\,\mathrm dx\\
    &=m\int_{\mathbb R}x\frac{\partial(\bar\psi^*\bar\psi)}{\partial t}\,\mathrm dx\\
    &=\frac{i\hbar}{2}\int_{\mathbb R}x\frac{\partial}{\partial x}\left( \bar\psi^*\frac{\partial\bar\psi}{\partial x}-\bar\psi\frac{\partial\bar\psi^*}{\partial x} \right)\,\mathrm dx\\
    &=\frac{i\hbar}{2}\left[x\bar\psi^*\frac{\partial\bar\psi}{\partial x}-x\bar\psi\frac{\partial\bar\psi^*}{\partial x}\right]_{-\infty}^\infty-\frac{i\hbar}{2}\int_{\mathbb R}\left( \bar\psi^*\frac{\partial\bar\psi}{\partial x}-\bar\psi\frac{\partial\bar\psi^*}{\partial x} \right)\,\mathrm dx\\
    &=-\frac{i\hbar}{2}\int_{\mathbb R}\left( \bar\psi^*\frac{\partial\bar\psi}{\partial x}-\bar\psi\frac{\partial\bar\psi^*}{\partial x} \right)\,\mathrm dx\\
    &=-i\hbar\int_{\mathbb R}\bar\psi^\ast\frac{\partial\bar\psi}{\partial x}\,\mathrm dx
\end{align*}
By a bad notation one can write
$$\langle p\rangle=\int_{\mathbb R}\psi^*\left( -i\hbar\frac{\partial}{\partial x} \right)\psi\,\mathrm dx$$
To make it even worse we can consider the functional $\hat{x}$ and $\hat{p}$ defined by $\hat{x}=x$ and $\hat{p}=-i\hbar\partial/\partial x$, which allows us to effectively confuse everybody by writing
$$\langle x\rangle=\int_{\mathbb R}\psi^*\hat{x}\psi\,\mathrm dx,\langle p\rangle=\int_{\mathbb R}\psi^*\hat{p}\psi\,\mathrm dx$$
In general dimensions, we can use the same idea with $\hat{\underline{x}}=\underline{x}$ and $\hat{\underline{p}}=-i\hbar\nabla$.
This is where we desperately try to justify these notations by considering them as (linear) operators in the Hilbert space $\mathcal H$, which works mathematically.
\footnote{However, the notation shall haunt you for a considerable proportion of your relationship with quantum mechanics.}
The kinetic energy operator is then
$$\hat{T}=\frac{\hat{p}^2}{2m}=-\frac{\hbar^2}{2m}\nabla^2$$
In fact, for any physical quantity $Q(x,p)$ , we are just gonna write $\hat{Q}=Q(\hat{x},\hat{p})$ and we can get
$$\langle Q(x,p)\rangle=\int_{\mathbb R}\psi^*\hat{Q}\psi\,\mathrm dx=\int_{\mathbb R}\psi^*Q\left( x,-i\hbar\frac{\partial}{\partial x} \right)\psi\,\mathrm dt$$
Also, one can check by direct calculation that
$$\frac{\mathrm d\langle p\rangle}{\mathrm dt}=\int_{\mathbb R}\psi^*\left( -\frac{\partial U}{\partial x} \right)\psi\,\mathrm dx=\langle -U_x\rangle$$
by using TDSE.
The Hamiltonian operator is $\hat{H}=\hat{T}+\hat{U}$, so
$$(\hat{H}\psi)(\underline{x},t)=-\frac{\hbar^2}{2m}\nabla^2\psi+U\psi$$
\subsection{Time-Independent Schr\"odinger Equation}
Note that we can rewrite the Schr\"odinger equation in the form
$$i\hbar\frac{\partial\psi}{\partial t}=\hat{H}\psi$$
SO if we seperate the variables $\psi(\underline{x},t)=X(\underline{x})T(t)$, then by some rearrangement
$$i\hbar T^{-1}T^\prime=(\hat{H}X)/X$$
The left hand side depends only on $t$ while the right hand side on $\underline{x}$, requiring them to be equal is just saying they are actually constants.
Denote this constant by $E$, then we get
$$\begin{cases}
    i\hbar T^{-1}T^\prime=E\\
    \hat{H}X=EX
\end{cases}$$
The first equation is easy to solve and gives $T(t)=e^{-iEt/\hbar}$.
The second equation is called the Time-Independent Schr\"odinger Equation (TISE).
Its solution $X$ is then interpreted as a physical state with energy $E$.
Note that $E$ must be real by our expression of $T$ and the condition that $T$ does not explode as $t\to\pm\infty$.
Also, TISE is essentially the eigenvalue problem of $\hat{H}$ which makes sense as it is a linear operator on the vector space of states and $0$.
So we have obtained the set of solutions $\psi=Xe^{-iEt/\hbar}$ where $X$ is an eigenfunction of $\hat{H}$ with eigenvalue $E$.
This set of solutions is called the stationary states.
\subsection{Stationary States}
For a stationary state, assuming $\psi$ is normalised, then
$$\rho(\underline{x},t)=|\psi(\underline{x},t)|^2=|X(\underline{x})|^2|e^{-iEt/\hbar}|^2=|X(\underline{x})|^2$$
So the stationary states are some particular solutions of TDSE whose induced probability distributions in space do not depend on time.
Of course, superpositions of this family of stationary states is also an allowed state (or zero).
What's more, if $\psi_1=X_1e^{-iE_1t/\hbar},\psi_2=X_2e^{-iE_2t/\hbar}$ are stationary states and $\psi=a_1\psi_1+a_2\psi_2$ is a superposition, then (assuming $a_i$ and $X_i$ are real),
\begin{align*}
    |\psi|^2&=(a_1^*X_1^*e^{iE_1t/\hbar}+a_2^*X_2^*e^{iE_2t/\hbar})(a_1X_1e^{-iE_1t/\hbar}+a_2X_2e^{-iE_2t/\hbar})\\
    &=|a_1|^2|X_1|^2+|a_2|^2|X_2|^2+a_1a_2X_1X_2(e^{i(E_1-E_2)t/\hbar}+e^{i(E_2-E_1)t/\hbar})\\
    &=|a_1|^2|X_1|^2+|a_2|^2|X_2|^2+2a_1a_2X_1X_2\cos\left( \frac{(E_1-E_2)t}{\hbar} \right)
\end{align*}
So $\psi$ is not a stationary state if $E_1\neq E_2$ as $|\psi|^2$ depends on time.
In fact, the stationary state is a basis of the vector space of states (and zero), that is each state $\psi$ can be expressed in the form
$$\psi(\underline{x},t)=\sum_{n=1}^Na_nX_n(x)e^{-iE_nt/\hbar}$$
where $X_n,E_n$ are stuff you expect them to be.
We then interpret $|a_n|^2$ to be the probability for the energy to be $E_n$.
\begin{remark}
    If we have a discrete and normalisable basis $X_n$ of the Hamiltonian $\hat{H}$.
    Then we might write something of the form
    $$\psi=\sum_{n=1}^\infty a_nX_ne^{-iE_nt/\hbar}$$
    which is a solution to TDSE if we can show that it converges nice enough.
    But this will require
    $$\lim_{R\to\infty}\int_{|\underline{x}|>R}|X_n|^2\,\mathrm dx=0$$
    So the particle cannot be too far from the origin.
    We call this a bounded state.\\
    How about a continuous basis (i.e. the basis is indexed by $(X_\alpha)_{\alpha\in I}$ where $I$ is an interval)?
    Then we might write
    $$\psi=\int_{\alpha\in I}A(\alpha)X_\alpha(\underline{x})e^{-iE_\alpha t/\hbar}\,\mathrm d\alpha$$
    So we interpret $|A(\alpha)|^2\,\mathrm d\alpha$ as the probability for the state to have energy $E_\alpha$.
    But the same limit condition does not have to hold, since even if the state themselves are not normalisable, we can choose an $A$ that decays fast enough to make the eventual superposition normalisable.
    This is called a scattering state.
\end{remark}
    \section{Solutions of TISE in One Dimension}
In one dimension, TISE restricts to
$$-\frac{\hbar^2}{2m}X^{\prime\prime}+UX=EX$$
where $E\in\mathbb R$.
\subsection{Infinte Potential Well}
Consider the potential
\footnote{Unorthodox, I know. I have stopped caring.}
$$U(x)=\begin{cases}
    0\text{, if $|x|\le a$}\\
    \infty\text{, if $|x|>a$}
\end{cases}$$
For $|x|>a$ we take the solution $X(x)=0$.
And we also want $X(\pm a)=0$ as we want $X$ to be continuous.
For $|x|\le a$, we are then aiming at the boundary value problem
$$-\frac{\hbar^2}{2m}X^{\prime\prime}=EX\iff X^{\prime\prime}+K^2X=0,K^2=\frac{2mE}{\hbar^2}\ge 0$$
subject to $X(\pm a)=0$.
This is known to have the general solution $X(x)=A\sin(Kx)+B\cos(Kx)$ for constants $A,B$.
The boundary conditions then require either $A=0$ and $K=n\pi/2a$ for $n=1,3,5,\ldots$ or $B=0$ and $K=n\pi/2a$ with $n=2,4,6,\ldots$.
So the allowed values of the energy would be
$$E_n=\frac{\hbar^2\pi^2}{8ma^2}n^2,n=1,2,3,\ldots$$
The lowest positive energy (aka ground state energy) is then $E_1=\hbar^2\pi^2/(8ma^2)$.
The solutions are
$$X_n(x)=\frac{1}{\sqrt{a}}\begin{cases}
    \cos(n\pi x/(2a))\text{, for $n=1,3,5,\ldots$}\\
    \sin(n\pi x/(2a))\text{, for $n=2,4,6,\ldots$}
\end{cases}$$
where the factor is obtained from the assumption that $X_n$ is normalised.
I am too lazy to plot the functions, but whoever read this are encouraged to plot a few of these states.
Without plotting, however, one can immediately realise that when $n$ is large, $X$ tends to fluctuate a lot.
The solution also allows us to draw the analogy between these solutions and standing waves with two endpoints fixed at $\pm a$.
Also, $X_n(-x)=(-1)^{n+1}X_n(x)$, so $X_n$ is either even or odd.
This is actually a general feature of the eigenfunctions of the Hamiltonian $\hat{H}$.
\begin{proposition}
    If $U$ is even and the energy spectrum is non-degenerate (i.e. we require $E_1<E_2<\cdots$), then each eigenfunction is either even or odd.
\end{proposition}
\begin{proof}
    If $U$ is even then TISE is reflection invariant, so $x\mapsto X(-x)$ is also a solution.
    But it also has energy $E$, so $X(-x)=AX(x)$ where $A$ is a constant.
    But then $X(x)=X(-(-x))=AX(-x)=A^2X(x)$, so $A=\pm 1$ as $X$ is not identically zero.
\end{proof}
\subsection{Finite Potential Well}
We consider the potential of the form
$$U(x)=\begin{cases}
    0\text{, if $|x|\le a$}\\
    U_0\text{, if $|x|>a$}
\end{cases}$$
where $U_0>0$.
We only consider the case $E<U_0$.
The other case will be dealt with later.
Also we only look for even $X$.
We define the nonnegative real quantities
$$K=\sqrt{\frac{2mE}{\hbar^2}},\kappa=\sqrt{\frac{2m(U_0-E)}{\hbar^2}}$$
For $|x|\le a$, we obtain the general solution $X(x)=A\sin(Kx)+B\cos(Kx)$ for $A,B$ constants.
We are looking for even states only, so we might as well write $X(x)=A\cos(Kx)$.
For $|x|>a$, we have $X^{\prime\prime}-\kappa^2X=0$ which has solution $X=Ce^{\kappa x}+De^{-\kappa x}$ for some constants $C,D$.
But the normalisation condition then tells us $C=0$ for $x>a$ and $D=0$ for $x<-a$.
Hence
$$X(x)=\begin{cases}
    Ce^{\kappa x}\text{, for $x<-a$}\\
    A\cos(Kx)\text{, for $|x|\le a$}\\
    De^{-\kappa x}\text{, for $x>a$}
\end{cases}$$
The continuity of $X$ and $X^\prime$ that we assumed then yield $K\tan(Ka)=\kappa$.
But the definition of $K$ and $\kappa$ yields $K\tan(Ka)=\kappa$.
We also know that $K^2+\kappa^2=2mU_0/\hbar^2$.
Define $\xi=Ka$ and $\eta=\kappa a$, then the two equations become
$$\begin{cases}
    \xi\tan\xi=\eta\\
    \eta^2+\eta^2=r_0^2
\end{cases}$$
where $r_0^2=2a^2mU_0/\hbar^2$.
By a plot, we know that there are only finitely many solutions $\{\xi_1,\ldots,\xi_p\}$ of $\xi$ with $(n-1)\pi\le\xi_n\le(n-1/2)\pi$ and the allowed energy are of the form
$$E_n=\frac{\hbar^2}{2ma^2}\xi_n^2,n=1,\ldots,r$$
As $U_0\to\infty$, we can get $r_0\to\infty$, so $p\to\infty$.
And the allowed energy would be
$$E_n=\frac{\hbar^2}{2ma^2}\xi_n^2=\frac{\hbar^2(2n-1)^2}{2ma^2}\pi^2$$
which is consistent with our result on infinite well.
Of course we can determine all the eigenfunctions as we have enough number of conditions on the breaking points and the normalisation condition.
One can also show by some calculation that the eigenfunctions also go to the eigenfunctions of the infinite well solution as $U_0\to\infty$.
\subsection{Free Particle}
If we require that
$$\int_{-\infty}^\infty |X(x)|^2\,\mathrm dx=N\in\mathbb R_+$$
then necessarily
$$\lim_{R\to\infty}\int_{|x|>R}|X(x)|^2\,\mathrm dx=0$$
Returning to our TISE $X^{\prime\prime}+K^2X(x)=0$ where $K=\sqrt{2mE/\hbar^2}$ which has solutions in the form $X_K(x)=Ae^{iKx}$.
This is obviously a continuous spectrum of solutions (corresponding to different values of $E$).
They gives the solutions $\psi_K(x,t)=X_k(x)e^{-i\hbar K^2t/(2m)}$.
But they are not in general normalisable!
Indeed,
$$\int_{\mathbb R}|\psi_K(x,t)|^2\,\mathrm dx=\int_{\mathbb R}|X_K(x,t)|^2\,\mathrm dx=|A|^2\int_{\mathbb R}\mathrm dx=\infty$$
unless $A=0$.
The workaround is to consider the integral superposition
$$\psi(x,t)=\int_{\mathbb R}A(K)\psi_K(x,t)\,\mathrm dK=\int_{\mathbb R}A(K)e^{iKx}e^{-i\hbar K^2t/(2m)}\,\mathrm dK$$
and choose $A$ that decreases fast enough as $K\to\infty$ so that $\psi$ is well-defined and normalisable.
A typical choice is the Gaussian wavepackage
$$A(K)=\exp\left( -\frac{\sigma}{2}(K-K_0)^2 \right)$$
where $\sigma>0$.
So using this we can write
$$\psi(x,t)=\int_{\mathbb R}\exp F(K)\,\mathrm dK$$
where
\begin{align*}
    F(K)&=-\frac{\sigma}{2}(K-K_0)^2+i\left(Kx-\frac{\hbar K^2}{2m}t\right)\\
    &=-\frac{1}{2}\left( \sigma+\frac{i\hbar t}{m} \right)K^2+(K_0\sigma+ix)K-\frac{\sigma}{2}K_0^2\\
    &=-\frac{\alpha}{2}\left( K-\frac{\beta}{\alpha} \right)^2+\frac{\beta^2}{2\alpha}+\delta,\alpha=\sigma+\frac{i\hbar t}{m},\beta=K_0\sigma+ix,\delta=-\frac{\sigma}{2}K_0^2
\end{align*}
Therefore,
\begin{align*}
    \psi(x,t)(x,t)&=\exp\left( \frac{\beta^2}{2\alpha}+\delta \right)\int_{-\infty}^\infty\exp\left( -\frac{\alpha}{2}\left( K-\frac{\beta}{\alpha} \right)^2 \right)\,\mathrm dK\\
    &=\exp\left( \frac{\beta^2}{2\alpha}+\delta \right)\int_{-\infty-i\nu}^{\infty-i\nu}\exp\left( -\frac{\alpha}{2}\tilde{K}^2 \right)\,\mathrm d\tilde{K},\tilde{K}=K-\frac{\beta}{\alpha},\nu=\operatorname{Im}\frac{\beta}{\alpha}\\
    &=\sqrt{\frac{2\pi}{\alpha}}\exp\left( \frac{\beta^2}{2\alpha}+\delta \right)
\end{align*}
I know the last step is staggering.
It's ok, you are stong enough to handle these traumas in your life.
Anyways, putting everything back gives
$$\psi(x,t)=\sqrt{\frac{2\pi}{\alpha}}\exp\left( -\frac{\sigma(x-\hbar K_0t/m)^2}{2(\sigma^2+\hbar^2t^2/m^2)} \right)$$
And (with the help of God) we can obtain the probability density
$$\rho(x,t)=\frac{C}{\sqrt{\sigma^2+\hbar^2t^2/m^2}}\exp\left( -\frac{\sigma(x-\hbar K_0t/m)^2}{\sigma^2+\hbar^2t^2/m^2} \right)$$
where $C$ is a normalisation constant, which can be determined by more calculation:
$$\int_{-\infty}^\infty\rho(x,t)\,\mathrm dx=1\implies C=\sqrt{\frac{\sigma}{\pi}}$$
Now, by a little calculation we obtain the mean $\langle x\rangle=\hbar K_0t/m$.
We write $v=\hbar K_0/m$ as an analogy of velocity.
The standard deviation is
$$\Delta x=\sqrt{\langle x^2\rangle-\langle x\rangle^2}=\sqrt{\frac{1}{2}\left( \sigma+\frac{\hbar^2t^2}{m^2\sigma} \right)}$$
which increases as $t\to\infty$, therefore the distribution is more spreaded out as $t\to\infty$.\\
Back to $\psi_K(x,t)=Ae^{iKx}e^{-i\hbar K^2t/(2m)}$, which as one can verify has standard deviation $\Delta x=\infty$ on position and $\Delta p=0$ on momentum since the momentum is just $p=\hbar K$.
These whole $\Delta x,\Delta p$ business will be dealt with later when we get to Heisenberg's Uncertainty Principle.
In fact, it is possible to show that the Gaussian wavepackage minimises uncertainty.
Now $\rho_K(x,t)=|A(K)|^2$ is then interpreted as the constant average density of a particle.
The probability current is
$$J_K(x,t)=-\frac{i\hbar}{2m}\left( \psi_K^*\frac{\partial\psi_K}{\partial x}-\psi_K\frac{\partial\psi_K^*}{\partial x} \right)=|A|^2\frac{\hbar K}{m}=|A|^2\frac{p}{m}$$
This is basically the product of the average density and velocity, which can be interpreted as the average flux of particles.
In some way, we are interpreting the state as a beam of particles where $A$ relates to the beam density.
\subsection{Scattering States}
Consider $U_0>0$ and
$$U(x)=\begin{cases}
    U_0\text{, for $x\in (0,a)$}\\
    0\text{, otherwise}
\end{cases}$$
Define
$$R=\lim_{t\to\infty}\int_{-\infty}^0|\psi(x,t)|^2\,\mathrm dx,T=\lim_{t\to\infty}\int_0^\infty|\psi(x,t)^2|\,\mathrm dx$$
If we interpret the system as a wave moving towards $x=0$ being scattered (partially reflected and partially transmitted), then $R$ can be seens as the reflection probability and $T$ the transmission probability.
If $\psi$ is normalised, then we have $R+T=1$.
Take the solutions $\psi_K(x,t)=X_K(x)e^{-i\hbar K^2t/(2m)}$ where $X_K(x)=Ae^{iKx}$.
\subsubsection{Scattering off a Potential Step}
Take $a\to\infty$ gives the potential
$$U(x)=\begin{cases}
    0\text{, for $x\le 0$}\\
    U_0\text{, for $x>0$}
\end{cases}$$
Restate the TISE
$$-\frac{\hbar^2}{2m}X^{\prime\prime}+UX=EX$$
First, we consider the case $E>U_0$.
For $x\le 0$, if we throw in $K=\sqrt{2mE/\hbar^2}$ the system gives
$$X_K(x)=X_K^{(+)}(x)+X_K^{(-)}(x),X_K^{(+)}(x)=Ae^{iKx},X_K^{(-)}(x)=Be^{-iKx}$$
where $A,B$ are constants.
Then we interpret $X_K^{(+)}$ as the beam of incident particles from $x=-\infty$ with $p=\hbar K$ and probability current $J_K^{(+)}=|A|^2\hbar K/m$.
The $X_K^{(-)}$ is correspondingly interpreted as the beam of reflected particles going towards $x=-\infty$ with $p=-\hbar k$ and $J_K^{(-)}=-|B|^2\hbar K/m$.
Now for $x>0$ we have $X^{\prime\prime}+\tilde{K}^2X=0$ where $\tilde{K}=\sqrt{2m(E-U_0)/\hbar^2}$ which is well-defined as we assumed $U_0<E$.
This yield the solutions
$$X_{\tilde{K}}=X_{\tilde{K}}^{(+)}+X_{\tilde{K}}^{(+)},X_{\tilde{K}}^{(+)}(x)=Ce^{i\tilde{K}x},X_{\tilde{K}}^{(-)}(x)=De^{-i\tilde{K}x}$$
Again $X_{\tilde{K}}^{(+)}$ is the beam of transmitted particles moving towards $x=+\infty$ with $p=\hbar\tilde{K}$ and $X_{\tilde{K}}^{(-)}$ is the beam if incident particles from $x=+\infty$ with $p=-\hbar \tilde{K}$.
We only want to consider the scattering problem, so we can set $D=0$.
Continuity of $X$ at $x=0$ gives $A+B=C$ and continuity of $X^\prime$ at $x=0$ gives $iKA-iKB=i\tilde{K}C$, therefore
$$B=\frac{K-\tilde{K}}{K+\tilde{K}}A,C=\frac{2K}{K+\tilde{K}}A$$
So the incident, reflective and transmitted probability current are
$$J_{\rm inc}=\frac{\hbar K}{m}|A|^2,J_{\rm ref}=\frac{\hbar K}{m}\left( \frac{K-\tilde{K}}{K+\tilde{K}} \right)^2|A|^2,J_{\rm tr}=\frac{\hbar \tilde{K}}{m}\frac{4K^2}{(K+\tilde{K})^2}|A|^2$$
Hence
$$R=\frac{J_{\rm ref}}{J_{\rm inc}}=\left( \frac{K-\tilde{K}}{K+\tilde{K}} \right)^2,T=\frac{J_{\rm tr}}{J_{\rm inc}}=\frac{4K\tilde{K}}{(K+\tilde{K})^2}$$
Easy to verify that $R+T=1$ and when $E\to \infty$ we have $K-\tilde{K}\to 0$ which implies $R\to 0,T\to 1$, which corresponds to the classical case.\\
For the case $E<U_0$, the equations become
$$\begin{cases}
    X^{\prime\prime}+K^2X=0\implies X=Ae^{iKx}+Be^{-iKx}\text{, for $x\le 0$}\\
    X^{\prime\prime}-\kappa^2X=0\implies X=Ge^{\kappa x}+Fe^{-\kappa x}\text{, for $x>0$}
\end{cases}$$
where as you expect,
$$K=\sqrt{\frac{2mE}{\hbar^2}},\kappa=\sqrt{\frac{2m(U_0-E)}{\hbar^2}}$$
We have $G=0$ because we expect the wavefunction to not blow up at $+\infty$.
Hence
$$X(x)=\begin{cases}
    Ae^{iKx}+B^{-iKx}\text{, for $x\le 0$}\\
    Fe^{-\kappa x}\text{, for $x>0$}
\end{cases}$$
Then the assumed continuity of $X,X^\prime$ at $0$ gives
$$B=\frac{iK+\kappa}{iK-\kappa}A,F=\frac{2iK}{iK-\kappa}A$$
So
$$j_{\rm tr}(x)=0,j_{\rm inc}=\frac{\hbar K}{m}|A|^2,j_{\rm ref}=\frac{\hbar K}{m}|B|^2=\frac{\hbar K}{m}|A|^2$$
Hence $R=1$ and $T=0$ just as we expect it.
\subsubsection{Scattering off a Potential Barrier}
Consider $a\in (0,\infty)$, so
$$U(x)=\begin{cases}
    U_0\text{, for $x\in (0,a)$}\\
    0\text{, otherwise}
\end{cases}$$
We deal with the $E\le U_0$ case first.
We define
$$K=\sqrt{\frac{2mE}{\hbar^2}},\bar{K}=\sqrt{\frac{2m(U_0-E)}{\hbar^2}}$$
Therefore
$$\begin{cases}
    X^{\prime\prime}(x)-\bar{K}^2X(x)=0\text{, for $x\in (0,a)$}\\
    X^{\prime\prime}(x)+K^2X(x)=0\text{, otherwise}
\end{cases}$$
which gives the general (normalised) solution
$$X(x)=\begin{cases}
    e^{iKx}+Ae^{-iKx}\text{, for $x\le 0$}\\
    Be^{-\bar{K}x}+Ce^{\bar{K}x}\text{, for $x\in (0,a)$}\\
    De^{iKx}+Fe^{-iKx}\text{, for $x\ge a$}
\end{cases}$$
We can set $F=0$ since we only consider the case where the incident beam is from $x=-\infty$.
By continuity of $X,X^\prime$ again we get
$$D=\frac{-4i\bar{K}K}{(\bar{K}-iK)^2e^{(\bar{K}+iK)a}-(\bar{K}+iK)^2e^{-(\bar{K}-iK)a}}$$
In classical dynamics, the particle can never pass the barrier.
But our result just shows that this is not in general the case, since
$$T=\frac{j_{\rm tr}}{j_{\rm inc}}=\frac{\hbar K|D|^2/m}{\hbar K/m}=|D|^2=\frac{4K^2\bar{K}^2}{(K^2+\bar{K}^2)^2\sinh^2(\bar{K}a)+4K^2\bar{K}^2}\neq 0$$
In particular, for $\bar{K}a>>1$, either $a>>1$ or $U_0>>E$ and
$$T\sim\frac{16K^2\bar{K}^2}{(K^2+\bar{K}^2)^2}e^{-2\bar{K}a}$$
So if the barrier is very wide or very tall, $T$ goes exponentially to $0$.
When $E<U_0$ but a beam of particles is transmitted, this phenomenon is called quantum tunneling.
\subsubsection{Physical Examples of Quantum Tunneling}
\begin{example}[Cold Emission]
    When light irradiates a metal surface, electrons are emitted when $\hbar\omega>W$ where $\omega$ is the angular frequency of photons and $W$ is a constant (called the work function) associated with the metal.
    An electron would be facing some sort of triangular barrier.
    In addition to the potential step at $x=a$ because of the work function $W$, suppose $\xi$ is the external electric field, then the resulting potential induced by the Lorentz force is $V_{\rm Lorentz}(x)=-e\xi(x-a)$.
    Their superposition would be a triangular barrier with cliff at $x=a$.
    So even for an electron with energy less than $W$, it can go to the part of the potential where it is low enough via quantum tunneling and excite from there.
    This is knowns as cold emission in photoelectric effect.
\end{example}
\begin{example}[Radioactive Decay]
    Consider the decay $\text{N}_{Z}^A\to \text{M}_{Z-2}^{A-4}+\text{He}_2^4$.
    For the $\alpha$ particle $\text{He}_2^4$, we want to know the potential field it is in.
    Indeed, if we let $r$ to be the distant between this $\alpha$ particle and the center of the nucleus, then $U(r)$ will begin negative near $0$ because of the attractive nuclear force and eventually go up positive again after $r$ gets large enough since the repulsive Coulumb force comes in play.
    Eventually, it goes down (but keep positive) and tends to $0$ as $r\to\infty$.
    This is known as the Gamow model.
    The curve certainly looks like a potential barrier.
    If we let $T$ be the transmission coefficient, then the half-life of this decay is actually proportional to $T^{-1}$.
\end{example}
\subsection{The Harmonic Oscillator}
The potential of the harmonic oscillator can be written as $U(x)=Kx^2/2=m\omega^2x^2/2$ where $K>0, \omega=\sqrt{k/m}>0$.
In classical mechanics, Newton's Second Law gives the equation $\ddot{x}=-\omega^2x$ which has solutions $x=A\sin(\omega t)+B\cos(\omega t)$ where $A,B$ are constants.
So the particle oscillates about $x=0$ with period $2\pi/\omega$.\\
In quantum mechanics, this gives the TISE
$$-\frac{\hbar^2}{2m}\frac{\mathrm d^2X}{\mathrm dx^2}+\frac{1}{2}m\omega^2x^2X(x)=EX(x)$$
where we expect to find a discrete set of normalisable eigenfunctions by looking at the shape of the potential.
As seen earlier, we can guarantee to find eigenfunctions that are either even or odd.
Write $\xi^2=m\omega x^2/\hbar$ and $\mathcal E=2E/(\hbar\omega)$, then this change of variables yield
$$-\frac{\mathrm d^2X}{\mathrm d\xi^2}+\xi^2X(\xi)=\mathcal EX(\xi)$$
For $\mathcal E=1$ it reduces to $-X^{\prime\prime}(\xi)+\xi^2X=X$.
Educated guess reveals that $X_0=A\exp(-\xi^2/2)$ is a family of solutions which is normalisable when $A\neq 0$.
This gives one pair of eigenvalue $E_0=\hbar\omega/2$ and eigenfunction $X_0(x)=A\exp(-m\omega x^2/(2\hbar))$ for $A\neq 0$.\\
For the general case, set $X(\xi)=f(\xi)\exp(-\xi^2/2)$, which transforms the equation into
$$-\frac{\mathrm d^2f}{\mathrm d\xi^2}+2\xi\frac{\mathrm df}{\mathrm d\xi}+(1-\mathcal E)f=0$$
Now we are desperate, so the obvious thing to do is to plug in the power series solution $f(\xi)=\sum_na_n\xi^n$ which gives
$$0=\sum_{n=0}^\infty [(n+1)(n+2)a_{n+2}-2na_n+(\mathcal E-1)a_n]\xi^n$$
which gives the recurrence
$$a_{n+2}=\frac{2n-\mathcal E+1}{(n+1)(n+2)}a_n$$
The recurrence steps by $2$, so it has even or odd solutions.
Here we need to prove an important claim
\begin{claim}
    Suppose the series does not terminate then $X(\xi)=f(\xi)\exp(-\xi^2/2)$ is not normalisable.
\end{claim}
\begin{proof}
    $a_{n+2}/a_n\sim 2/n$ as $n\to\infty$, which means that $f(\xi)\sim\exp(\xi^2)$, hence $X$ is not normalisable.
\end{proof}
Therefore, for a normalisable solution to exist, $\mathcal E$ must be an odd positive integer, so we get the eigenvalues and eigenfunctions
$$E_N=\left( N+\frac{1}{2} \right)\hbar\omega,X_N(x)=f_N\left( \sqrt{\frac{m\omega}{\hbar}} \right)\exp\left( -\frac{m\omega x^2}{2\hbar} \right)$$
where $f_N$ are the respective polynomials that yield from the series solution as the choice of $\mathcal E$ makes it terminate.
These are called Hermite polynomials which, as one can verify, satisfies $f_N(-\xi)=(-1)^Nf_N(\xi)$ and
$$f_N(\xi)=(-1)^n\exp(\xi^2)\frac{\mathrm d^n}{\mathrm d\xi^n}\exp(-\xi^2)$$
We've got nothing better to do so why not calculate some of them.
\begin{center}
    \begin{tabular}{c|c|c|c}
        $N$&$f_N(\xi)$&$E_N$&$X_N(\xi)$\\ \hline
        $0$&$1$&$\hbar\omega/2$&$\exp(-\xi^2/2)$\\
        $1$&$\xi$&$3\hbar\omega/2$&$\xi\exp(-\xi^2/2)$\\
        $2$&$1-2\xi^2$&$5\hbar\omega/2$&$(1-2\xi^2)\exp(-\xi^2/2)$\\
        $3$&$\xi-2\xi^3/3$&$7\hbar\omega/2$&$(\xi-2\xi^3/3)\exp(-\xi^2/2)$
    \end{tabular}
\end{center}
    \section{Axioms for Quantum Mechanics}
Now we start to try and make the theory more mathematical.
\subsection{Vector Space and Inner Product}
As we have seen, we can collect $0$ and the set of possible wavefunctions (for fixed time $t$) as a vector space $\mathcal H$ which is a subspace of $L^2(\mathbb R^3,\mathbb C)$.
Linear maps involved are operators and the inner product is what we have seen earlier:
\begin{definition}
    The inner product of two wavefunctions $\psi,\phi$ is defined by
    $$(\psi,\phi)=\int_{\mathbb R^3}\psi^*\phi\,\mathrm dV$$
\end{definition}
\begin{lemma}
    If $\psi,\phi\in L^2(\mathbb R^3,\mathbb C)$, then $(\psi,\phi)$ exists.
\end{lemma}
\begin{proof}
    Cauchy-Schwartz.
\end{proof}
\begin{proposition}
    1. $(\psi,\phi)=(\phi,\psi)^*$\\
    2. $(\psi,\lambda_1\phi_1+\lambda_2\phi_2)=\lambda_1(\psi,\phi_1)+\lambda_2(\psi,\phi_2)$ and $(\mu_1\psi_1+\mu_2\psi_2,\phi)=\mu_1^*(\psi_1,\phi)+\mu_2^*(\psi_2,\phi)$.\\
    3. $( \psi,\psi)\ge 0$ and equals $0$ iff $\psi=0$.
\end{proposition}
\begin{proof}
    Obvious.
\end{proof}
\begin{definition}
    A wavefunction $\psi$ is normalised if $(\psi,\psi)=1$.
    If $(\psi,\phi)=0$ then we say $\psi,\phi$ are orthogonal.
    A set of wavefunctions $\{\psi_n\}$ is orthogonal if $(\psi_m,\psi_n)=\delta_{mn}$.
    It is complete if an other wavefunction $\phi$ can be written as
    $$\phi=\sum_{n=0}^\infty c_n\psi_n$$
    for some $c_n\in\mathbb C$.
\end{definition}
It then follows that
\begin{proposition}
    If we have a complete and orthogonal set of wavefunctions $\{\psi_n\}$ and $\phi=\sum_n c_n\psi_n$, then $c_n=(\psi_n,\phi)$.
\end{proposition}
\begin{proof}
    Straight from definition.
\end{proof}
The probabilistic interpretation of inner product is that if we take $\phi$ to be the desired outcome of a measurement and $\psi$ is the actual state of the particle at the time of measurement, then $|( \psi,\phi)|^2$ can be taken as the probability of measuring $\phi$ as an outcome, that is the overlap between the two states at time $t$.
\subsection{Hermitian Operators}
An operator on $\mathcal H$, the vector space of possible wavefunctions and $0$, is a linear map from $\mathcal H$ to itself (or a vector space containing it).
\begin{example}
    Linear differential and translation are operators.
    We can also have the parity operator $\phi(x)\mapsto\psi(-x)$.
\end{example}
\begin{definition}
    The Hemitian conjugate $\hat{A}^\dagger$ of an operator $\hat{A}$ is an operator such that $(\hat{A}^\dagger\psi_1,\psi_2)=(\psi_1,\hat{A}\psi_2)$ for any $\psi_1,\psi_2\in\mathcal H$.
\end{definition}
Consequently, $(a_1\hat{A}_1+a_2\hat{A}_2)^\dagger=a_1^*\hat{A}^\dagger+a_2^*\hat{A}_2^\dagger$ and $(\hat{A}\hat{B})^\dagger=\hat{B}^\dagger\hat{A}^\dagger$.
\footnote{Yes, we haven't proved the existence and uniqueness of a conjugate. And yes, the lecturer still hasn't proved it.}
\begin{definition}
    A Hermitian operator is an operator $\hat{A}$ such that $(\hat{A}\psi_1,\psi_2)=(\psi_1,\hat{A}\psi_2)$.
\end{definition}
Equivalently $\hat{A}=\hat{A}^\dagger$.
\begin{example}
    Physical quantities (observables) like $\hat{x},\hat{p},\hat{H}$ are Hermitian operators.
\end{example}
\begin{theorem}
    The eigenvalues of Hermitian operators are real.
\end{theorem}
\begin{proof}
    Let $\hat{A}$ be Hermitian and $\psi$ a normalised eigenfunction of it with eigenvalue $a$, that is $\hat{A}\psi=a\psi$.
    Now
    \begin{align*}
        a&=a(\psi,\psi)=( \psi,a\psi)=(\psi,\hat{A}\psi)=( \hat{A}^\dagger\psi,\psi)=(\hat{A}\psi,\psi)\\
        &=( a\psi,\psi)=a^*(\psi,\psi)\\
        &=a^*
    \end{align*}
    So $a$ has to be real.
\end{proof}
\begin{theorem}
    Let $\hat{A}$ be a Hermitian operator and $\psi_1,\psi_2$ normalised eigenfunctions with different eigenvalues $a_1\neq a_2$, then $\psi_1,\psi_2$ are orthogonal.
\end{theorem}
\begin{proof}
    We have
    \begin{align*}
        a_1(\psi_1,\psi_2 )&=a_1^*(\psi_1,\psi_2)=( a_1\psi_1,\psi_2)=( \hat{A}\psi_1,\psi_2)\\
        &=( \hat{A}^\dagger\psi_1,\psi_2)=(\psi_1,\hat{A}\psi_2)=(\psi_1,a_2\psi_2)\\
        &=a_2(\psi_1,\psi_2)
    \end{align*}
    But $a_1\neq a_2$, so it has to be the case that $(\psi_1,\psi_2)=0$.
\end{proof}
\begin{theorem}
    The set of eigenfunctions, discrete or continous, of any Hermitian operator forms a complete orthogonal set.
\end{theorem}
\begin{proof}
    Haha.
\end{proof}
\begin{corollary}
    Every solution of TDSE can be written as a superposition of the stationary states.
\end{corollary}
\begin{proof}
    The stationary states are eigenfunctions of the Hamiltonian.
\end{proof}
\subsection{Quantum Measurements}
We have a few postulates for quantum mechanics.
\begin{postulate}
    1. Any quantum observable $O$ is represented by a Hermitian operator $\hat{O}$.\\
    2. The possible outcomes of measurement of $O$ are eigenvalues of $\hat{O}$.\\
    3. if $\hat{O}$ has a discrete set $\{\psi_i\}$ of normalised eigenfunctions with distinct eigenvalues $\{\lambda_i\}$, then the measurement of $O$ on a particle in state $\psi=\sum_ia_i\phi_i$, then the probability of outcome $\lambda$ is $p(\lambda_i)=|a_i|^2$.
    In particular, if $\psi=\psi_i$, then the measurement is $\lambda_i$ with probability $1$.\\
    4. If $\{\psi_i\}_{i\in I}$ is a subset of the eigenfunctions with common eigenvalue $\lambda$, then $p(\lambda)=\sum_{i\in I}|a_i|^2$.\\
    5. (consequence of 3 and 4) $\sum_i|a_i|^2=1$.\\
    6. (projection postulate, collapse of wavefunction) If $O$ is measured on $\psi$ at time $t$ and outcome of the measure is $\lambda_i$, then the wavefunction instaneously becomes $\psi_1$ at time $t$.\\
    7. If $\hat{O}$ has an eigenvalue $\lambda$ with eigenfunctions $\{\psi_i\}_{i\in I}$, then if $O$ is measured with outcome $\lambda$ the wavefunction instaneously becomes a superposition of $\{\psi_i\}_{i\in I}$.
\end{postulate}
\begin{definition}
    The projection operator $\hat{p}_i$ sends $\psi=\sum_ja_j\psi_j$ to $a_i\psi_i$.
\end{definition}
\subsection{Expected Values}
Consider the measurement of observable $O$ on state $\psi$ and the corresponding Hermitian operator $\hat{O}$ has a discrete set of eigenfunctions $\{\psi_i\}$ with eigenvalues $\lambda_i$, then $\{\lambda_i\}$ gives us all possible outcomes of measures of $O$ with $p(\lambda_i)=|\langle\psi,\psi_i\rangle|^2$.
So we can defined\begin{definition}
    The expected value of $O$ is
    $$\langle\hat{O}\rangle_\psi=\sum_ip_i\lambda_i=\sum_i|(\psi,\psi_i)|^2\lambda_i=(\psi,\hat{O}\psi)=\int_{\mathbb R^3}\psi^*\hat{O}\psi\,\mathrm dV$$
\end{definition}
Then easily $\langle\cdot\rangle_\psi$ is a linear form on the (real) vector space of Hermitian operators on $\mathcal H$.
\subsection{Commutators}
\begin{definition}
    The commutator of two operators $\hat{A},\hat{B}$ is the operator $[\hat{A},\hat{B}]=\hat{A}\hat{B}-\hat{B}\hat{A}$.
    Correspondingly, the anti-commutator is $\{\hat{A},\hat{B}\}=\hat{A}\hat{B}+\hat{B}\hat{A}$
\end{definition}
Easly $[\cdot,\cdot]$ is a antisymmetric and
$$[\hat{A},\hat{B}\hat{C}]=[\hat{A},\hat{B}]\hat{C}+\hat{B}[\hat{A},\hat{C}],[\hat{A}\hat{B},\hat{C}]=\hat{A}[\hat{B},\hat{C}]+[\hat{A},\hat{C}]\hat{B}$$
\begin{example}
    We have $[\hat{x},\hat{p}]=i\hbar\hat{I}$ where $\hat{I}$ is the identity operator.
\end{example}
\begin{definition}
    Two Hemitian operators $\hat{A},\hat{B}$ are simultaneously diagonalisable in $\mathcal H$ if there is a complete basis of joint eigenfunctions $\{\psi_i\}$ with $\hat{A}\psi_i=a_i\psi_i$ and $\hat{B}\psi_i=b_i\psi_i$ for all $i$ where $a_i,b_i$ are the respective eigenvalues.
\end{definition}
\begin{theorem}
    Two Hemitian operators $\hat{A}$ and $\hat{B}$ are simultaneously diagonalisable iff $[\hat{A},\hat{B}]=0$
\end{theorem}
\begin{proof}
    The ``only if'' direction is trivial.
    Conversely, if $[\hat{A},\hat{B}]=0$, then $\hat{A}\hat{B}=\hat{B}\hat{A}$, so $A(B\psi_i)=a_i(\hat{B}\psi_i)$ for all $i$.
    This means that $\hat{B}$ maps any eigenspace $E$ of $\hat{A}$ to itself.
    But $\hat{B}$ is also Hermitian on $E$, so we can have an eigenspace $E$ in which $\hat{B}$ acts diagonally.
    Collect them together gives the desired complete set of basis.
\end{proof}
\subsection{Heisenberg's Uncertainty Principle}
\begin{definition}
    The uncertainty in mass of $A$ on a state $\psi$ is defined as
    $$(\Delta_\psi A)^2=\langle (\hat{A}-\langle\hat{A}\rangle_\psi\hat{I})^2\rangle_\psi=\langle\hat{A}^2\rangle_\psi-(\langle\hat{A}\rangle_\psi)^2$$
    where $\hat{I}$ is the identity operator.
\end{definition}
\begin{lemma}
    $(\Delta_\psi A)^2\ge 0$ and $\Delta_\psi A=0$ iff $\psi$ is an eigenfunction of $\hat{A}$.
\end{lemma}
\begin{proof}
    Write $\phi=(\hat{A}-\langle\hat{A}\rangle_\psi\hat{I})\psi$, then
    \begin{align*}
        (\Delta_\psi A)^2&=\langle (\hat{A}-\langle\hat{A}\rangle_\psi\hat{I})^2\rangle_\psi\\
        &=((\hat{A}-\langle\hat{A}\rangle_\psi\hat{I})\psi,(\hat{A}-\langle\hat{A}\rangle_\psi\hat{I})\psi)\\
        &=(\phi,\phi)\ge 0
    \end{align*}
    and equality holds iff $\phi=0$ which happens iff $\phi=0$, but this is just another way of saying $\phi$ is an eigenfunction of $\hat{A}$.
\end{proof}
\begin{theorem}[Schwartz Inequality]
    If $\phi,\psi$ are any two normalisable wavefunctions, then $|(\phi,\psi)|^2\le(\phi,\phi)(\psi,\psi)$ with equality iff $\phi,\psi$ are linearly dependent.
\end{theorem}
\begin{proof}
    Just copy the proof of the usual Cauchy-Schwartz Inequality.
\end{proof}
So we can safely write $\Delta_\psi A=\sqrt{(\Delta_\psi A)^2}$.
\begin{theorem}[Generalised Uncertainty Theorem]
    If $A,B$ are observables and $\phi\in\mathcal H$, then
    $$(\Delta_\psi A)(\Delta_\psi B)\ge\frac{1}{2}|(\psi,[\hat{A},\hat{B}]\psi)|$$
\end{theorem}
\begin{proof}
    Write $\hat{A}'=\hat{A}-\langle\hat{A}\rangle_\psi\hat{I}$ which is also Hermitian, then we have $(\Delta_\psi A)^2=(\hat{A}'\psi,\hat{A}'\psi)$.
    Similarly $\hat{B}'=\hat{B}-\langle\hat{B}\rangle_\psi\hat{I}$ gives $(\Delta_\psi B)^2=(\hat{B}'\psi,\hat{B}'\psi)$.
    Note that we have $[\hat{A}',\hat{B}']=[\hat{A},\hat{B}]$.
    So Schwartz Inequality gives
    $$(\Delta_\psi A)^2(\Delta_\psi B)^2=(\hat{A}'\psi,\hat{A}'\psi)(\hat{B}'\psi,\hat{B}'\psi)\ge|(\hat{A}'\psi,\hat{B}'\psi)|^2=|(\psi,\hat{A}'\hat{B}'\psi)|^2$$
    But we can write $\hat{A}'\hat{B}'=([\hat{A}',\hat{B}']+\{\hat{A}',\hat{B}'\})/2$.
    Consequently,
    $$(\Delta_\psi A)^2(\Delta_\psi B)^2\ge \frac{1}{4}|(\psi,[\hat{A}',\hat{B}']\psi)+(\psi,\{\hat{A}',\hat{B}'\}\psi) |^2$$
    But
    $$(\psi,\{\hat{A}',\hat{B}'\}\psi)=(\{\hat{A}',\hat{B}'\}^\dagger\psi,\psi)=(\{\hat{A}',\hat{B}'\}\psi,\psi)=(\psi,\{\hat{A}',\hat{B}'\}\psi)^*$$
    so $(\psi,\{\hat{A}',\hat{B}'\}\psi)$ has to be real.
    Similarly $(\psi,[\hat{A}',\hat{B}']\psi)$ has to be purely imaginary.
    Therefore
    \begin{align*}
        (\Delta_\psi A)^2(\Delta_\psi B)^2&\ge \frac{1}{4}|(\psi,[\hat{A}',\hat{B}']\psi)+(\psi,\{\hat{A}',\hat{B}'\}\psi) |^2\\
        &=\frac{1}{4}|(\psi,[\hat{A}',\hat{B}']\psi)|^2+|(\psi,\{\hat{A}',\hat{B}'\}\psi) |^2\\
        &\ge\frac{1}{4}|(\psi,[\hat{A}',\hat{B}']\psi)|^2\\
        &=\frac{1}{4}|(\psi,[\hat{A},\hat{B}]\psi)|^2
    \end{align*}
    Taking square root on both sides shows the theorem.
\end{proof}
If $[\hat{A},\hat{B}]=0$, then the bound is just zero.
This is interpreted as we can measure $A,B$ simultaneously.
\begin{corollary}[Heisenberg's Uncertainty Principle]
    $$(\Delta_\psi x)(\Delta_\psi p)\ge\frac{\hbar}{2}$$
\end{corollary}
\begin{proof}
    Just take $\hat{A}=\hat{x},\hat{B}=\hat{p}$ and be reminded that $[\hat{x},\hat{p}]=i\hbar\hat{I}$.
\end{proof}
\begin{example}
    1. We can see a particle with light of wavelength $\lambda\sim\Delta x$ around its de Broglie wavelength $h/p$, so $\Delta p\sim p\sim\hbar/(\Delta x)$ which means $\Delta x\Delta p\sim h=2\pi\hbar>\hbar/2$.\\
    2. Consider the unnormalisable plane wave solution $\psi_p=e^{ipx/\hbar}$ for a free particle, then $\Delta_{\psi_p}p=0,\Delta_{\psi_p}x=\infty$.
    Recall that we need to make a certain superposition of it normalisable by introducing the Gaussian wavepackage
    $$\psi_{\rm GP}(x,t)=\sqrt[4]{\frac{\sigma}{\pi(\sigma^2+\hbar^2t^2/m^2)}}\exp\left( -\frac{\sigma(x-\hbar k_0t/m)^2}{2(\sigma^2+\hbar^2t^2/m^2)} \right)$$
    where we actually have $(\Delta_{\psi_{\rm GP}}x)(\Delta_{\psi_{\rm GP}}p)=\hbar/2$.
\end{example}
We can obtain the equality in the second example by calculation of course, but the gist is actually the following:
\begin{lemma}
    If $\hat{x}\psi=ia\hat{p}\psi$ for some $a\in\mathbb R$, then $(\Delta_\psi x)(\Delta_\psi p)=\hbar/2$.
\end{lemma}
This condition is easily seen to be necessary.
\begin{proof}
    The condition shows that we have the equality case in Schwartz inequality in the proof of the preceding theorem.
    Furthermore,
    \begin{align*}
        (\psi,\{\hat{x},\hat{p}\}\psi)&=(\psi,\hat{x}\hat{p}\psi)+(\psi,\hat{p}\hat{x}\psi)\\
        &=(\hat{x}\psi,\hat{p}\psi)+(\hat{p}\psi,\hat{x}\psi)\\
        &=(ia\hat{p}\psi,\hat{p}\psi)+(\hat{p}\psi,ia\hat{p}\psi)\\
        &=(-ia+ia)(\hat{p}\psi,\hat{p}\psi)\\
        &=0
    \end{align*}
    which implies the lemma.
\end{proof}
\begin{lemma}
    $\hat{x}\psi=ia\hat{p}\psi$ iff $\psi(x)\propto e^{-bx^2}$ for some $b>0$.
\end{lemma}
\begin{proof}
    Obvious.
\end{proof}
\subsection{Ehrenfest's Theorem}
\begin{theorem}[Ehrenfest's Theorem]
    Let $\hat{A}$ be an operator, then
    $$\frac{\mathrm d}{\mathrm dt}\langle\hat{A}\rangle_\psi=\frac{i}{\hbar}\langle [\hat{H},\hat{A}]\rangle_\psi+\langle\partial\hat{A}/\partial t\rangle_\psi$$
\end{theorem}
\begin{proof}
    Just expand.
    \begin{align*}
        \frac{\mathrm d}{\mathrm dt}\langle\hat{A}\rangle_\psi&=\frac{\mathrm d}{\mathrm dt}\int_{-\infty}^\infty\psi^*\hat{A}\psi\,\mathrm dx\\
        &=\int_{-\infty}^\infty\frac{\mathrm d}{\mathrm dt}(\psi^*\hat{A}\psi)\,\mathrm dx\\
        &=\int_{-\infty}^\infty \left( \frac{\partial\psi^*}{\partial t}\hat{A}\psi+\psi^*\hat{A}\frac{\partial\psi}{\partial t} \right)\,\mathrm dx+\langle\partial\hat{A}/\partial t\rangle_\psi\\
        &=\frac{i}{\hbar}\int_{-\infty}^\infty (\psi^*\hat{H}\hat{A}\psi-\psi^*\hat{A}\hat{H}\psi)\,\mathrm dx+\langle\partial\hat{A}/\partial t\rangle_\psi\\
        &=\frac{i}{\hbar}\langle[\hat{H},\hat{A}]\rangle_\psi+\langle\partial\hat{A}/\partial t\rangle_\psi
    \end{align*}
    which is what we wanted.
\end{proof}
\begin{example}
    Take $\hat{A}=\hat{H}$, then $[\hat{H},\hat{H}]=0$, so $\mathrm d\langle\hat{H}\rangle_\psi/\mathrm dt=0$ which is equivalent to the conservation of total energy in quantum mechanics.\\
    Take $\hat{A}=\hat{p}$, then $[\hat{H},\hat{p}]=i\hbar\partial U/\partial x$, therefore,
    $$\frac{\mathrm d\langle\hat{p}\rangle_\psi}{\mathrm dt}=-\left\langle\frac{\mathrm dU}{\mathrm dx}\right\rangle_\psi$$
    which is analogous to the classical case.\\
    Take $\hat{A}=\hat{x}$, ten we obtain $[\hat{H},\hat{x}]=-i\hbar\hat{p}/m$,
    $$\frac{\mathrm d\langle\hat{x}\rangle_\psi}{\mathrm dt}=\frac{\langle\hat{p}\rangle_\psi}{m}$$
    which is again analogous to classical mechanics.
\end{example}
\subsection{The Harmonic Oscillator Revisited}
For a harmonic oscillator, the Hamiltonian is
\begin{align*}
    \hat{H}&=\frac{\hat{p}^2}{2m}+\frac{1}{2}m\omega^2\hat{x}^2\\
    &=\frac{1}{2m}(\hat{p}+im\omega\hat{x})(\hat{p}-im\omega\hat{x})+\frac{i\omega}{2}[\hat{p},\hat{x}]\\
    &=\frac{1}{2m}(\hat{p}+im\omega\hat{x})(\hat{p}-im\omega\hat{x})+\frac{\hbar\omega}{2}
\end{align*}
Write $\hat{a}=(\hat{p}-im\omega\hat{x})/\sqrt{2m}$ (called the ladder operator), then $\hat{a}^\dagger=(\hat{p}+im\omega\hat{x})/\sqrt{2m}$ and $\hat{H}=\hat{a}^\top\hat{a}+\hbar\omega/2$.
Also $[\hat{a},\hat{a}^\dagger]=\hbar\omega\hat{I}$, therefore $\hat{a}$ is Hermitian.
In addition $[\hat{H},\hat{a}]=-\hbar\omega\hat{a}$ and $[\hat{H},\hat{a}^\dagger]=\hbar\omega\hat{a}^\dagger$.
Suppose $X$ is an eigenfunction of $\hat{H}$ with eigenvalue $E$, that is $\hat{H}X=EX$, then $\hat{H}\hat{a}X=[\hat{H},\hat{a}]X+\hat{a}\hat{H}X=(-\hbar\omega+E)\hat{a}X$.
Similarly $\hat{a}^\dagger =(E+\hbar\omega)\hat{a}^\dagger X$.
Then by induction $\hat{a}^nX$ is an eigenfunction of $\hat{H}$ with eigenvalue $E-n\hbar\omega$ and $\hat{a}^\dagger X$ an eigenfunction with eigenvalue $E+n\hbar\omega$.
For $U\ge 0$, we have $\langle H\rangle_\psi\ge 0$, so we can choose the lowest positive eigenfunction $X_0$ and we must then have $\hat{a}X_0=0$.
This is just a first order differential equation, which can be easily solved to give $X_0(x)\propto\exp(-m\omega x^2/(2\hbar))$.
So $X_0$ has eigenvalue $\hbar\omega/2$.
To find the excited states, we simply need to compute
$$X_n=(\hat{a}^\dagger)^nX_0\propto\frac{1}{\sqrt{2m}}(\hat{p}+im\omega\hat{x})^n\exp(-m\omega x^2/(2\hbar))$$
which involves Hermite polynomials.
They then have eigenvalue $E_n=(n+1/2)\hbar\omega$.
    \section{Schr\"odinger's Equation in Three Dimensions}
\subsection{TISE in 3D for Spherically Symmetric Potentials}
The TISE in 3D is in the form $\hat{H}X=EX$ where
$$\hat{H}=-\frac{\hbar^2}{2m}\nabla^2X+UX$$
In spherical coordiates $x=r\cos\phi\cos\theta,y=r\sin\phi\sin\theta,z=r\cos\theta$ where $r\in[0,\infty],\theta\in[0,\pi],\phi\in[0,2\pi)$, the Laplacian becomes
$$\nabla^2X=\frac{1}{r}\frac{\partial^2(rX)}{\partial r^2}+\frac{1}{r^2\sin^2\theta}\left( \sin\theta\frac{\partial}{\partial\theta}\left( \sin\theta\frac{\partial X}{\partial\theta} \right)+\frac{\partial^2X}{\partial\phi^2} \right)$$
We consider the case where $U=U(r,\theta,\phi)=U(r)$ and we are only looking for spherically symmetric solutions $X(r,\theta,\phi)=X(r)$.
Then the Hamiltonian becomes
$$\hat{H}X=-\frac{\hbar^2}{2m}\frac{1}{r}\frac{\mathrm d^2(rX)}{\mathrm dr^2}+UX=-\frac{\hbar^2}{2m}\left( \frac{\mathrm d^2X}{\mathrm dr^2}+\frac{2}{r}\frac{\mathrm dX}{\mathrm dr} \right)+UX$$
We have the normalisation condition
$$\int_{\mathbb R^3}|\psi^2|\,\mathrm dV<\infty\implies \int_0^\infty|X(r)|^2r^2\,\mathrm dr<\infty$$
which means $X(r)\to 0$ sufficiently fast as $r\to\infty$.\\
Consider $\sigma(r)=rX(r)$, then TISE becomes $\hat{H}\sigma=E\sigma$ on the half-plane $r\ge 0$.
Now the plan is to solve this on the whole line with $U(-r)=U(r)$.
As $X(0)$ are defined, $\sigma(0)=0$.
We only need to look for odd solutions $\sigma(-r)=-\sigma(r)$ since:
\begin{lemma}
    There is no nontrivial even solution to the problem.
\end{lemma}
\begin{proof}
    Suppose there is, say it is $\sigma_+$, then necessarily it is $C^2$ (for the equation to make sense) and hence $\sigma_+^\prime(0)=\sigma_+^{\prime\prime}(0)=0$.
    Then we have a corresponding odd solution
    $$\sigma_-(r)=\begin{cases}
        \sigma_+(r)\text{, for $r\ge 0$}\\
        -\sigma_+(r)\text{, for $r<0$}
    \end{cases}$$
    But then $\sigma(r)=\sigma_+(r)-\sigma_-(r)$ is a solution to the equation by linearity and vanishes for any $r>0$, which should not happen.
    \footnote{Well.}
\end{proof}
\begin{example}
    Consider the symmetric potential well
    $$U(r)=\begin{cases}
        0\text{, for $r\le a$}\\
        U_0\text{, for $r>a$}
    \end{cases}$$
    where $a,U_0>0$.
    For $0<E<U_0$, again let $k=\sqrt{2mE/\hbar^2}$ and $\bar{k}=\sqrt{2m(U_0-E)/\hbar^2}$, so
    $$\sigma(r)=\begin{cases}
        A\sin(kr)\text{, for $r\le a$}\\
        Be^{-\bar{k}r}\text{, for $r<a$}
    \end{cases}$$
    The boundary conditions gives $-k\cot(ka)=\bar{k}$ and we already know that $k^2+\bar{k}^2=2mU_0/\hbar^2$.
    Again define $\xi=ka,\eta=\bar{k}a$ and $r_0=a\sqrt{2mU_0/\hbar^2}$, so we have
    $$\begin{cases}
        -\xi\cot(\xi)=\eta\\
        \xi^2+\eta^2=r_0^2
    \end{cases}$$
    Now if $r_0<\pi/2$ (that is $U_0<\pi^2\hbar^2/(8ma^2)$), then there is no bounded solution.
    But unlike the one-dimensional case, we can always find at least one bounded state.
\end{example}
\subsection{The Radial Wavefunction of Hydrogen Atom}
The hydrogen atom consists of a proton $p^+$ and an electron $e^-$.
The Coulomb force is
$$F=-\frac{e^2}{4\pi\epsilon_0r^2}=-\frac{\partial U}{\partial r}\implies U(r)=-\frac{e^2}{4\pi\epsilon_0r}$$
So the bound states must have $E<0$, and $e^-$ is at rest $E=0$ as $r\to\infty$.
We want to look for stationary states of $e^-$ with spherically symmetric wavefunction $X=X(r)$ to
$$-\frac{\hbar^2}{2m}\left( \frac{\mathrm d^2X}{\mathrm dr^2}+\frac{2}{r}\frac{\mathrm dX}{\mathrm dr} \right)-\frac{e^2}{4\pi\epsilon_0r}X=EX$$
Then we need $E<0$ for bound states.
Define
$$\nu=\sqrt{-\frac{2mE}{\hbar^2}},\beta=\frac{e^2m}{2\pi\epsilon_0\hbar^2}$$
Then the equation becomes
$$\frac{\mathrm d^2X}{\mathrm dr^2}+\frac{2}{r}\frac{\mathrm dX}{\mathrm dr}+\left( \frac{\beta}{r}-\nu^2 \right)X=0$$
As $r\to\infty$, if we assume $X,X^\prime$ do not explode near $\infty$, then the equation becomes $X^{\prime\prime}-\nu^2X=0$, so $X(r)\sim e^{-\nu r}$ as $r\to\infty$.
This inspires us to try $X(r)=f(r)e^{-\nu r}$, which transforms the equation into
$$\frac{\mathrm d^2f}{\mathrm dr^2}+\frac{2}{r}(1-\nu r)\frac{\mathrm df}{\mathrm dr}+\frac{1}{r}(\beta-2\nu)f=0$$
which has a regular singular point at $r=0$, so we shall try the series solution $f(r)=r^c\sum_n a_nr^n$.
Plugging in,
$$0=\sum_{r=0}^\infty a_n(c+n)(c+n-1)r^{c+n-2}+\frac{2}{r}(1-\nu r)a_n(c+n)r^{c+n-1}+(\beta-2\nu)r^{c+n-1}$$
By looking at the coefficient of $r^{c-2}$, we have $0=a_0c(c+1)r^{c-2}$, so $c=-1$ or $c=0$.
The former is discarded as it has a singularity at $0$.
So $c=0$, hence $f(r)=\sum_na_nr^n$ and
$$0=\sum_{n=2}^\infty (a_nn(n+1)+a_{n-1}(\beta-2\nu))r^{n-2}=0\implies a_n=\frac{2\nu n-\beta}{n(n+1)}a_{n-1}$$
So all the coefficients are determined from $a_0$.
\begin{lemma}
    If the series above does not terminate, then the function $X(r)=f(r)e^{-\nu r}$ is not normalisable.
\end{lemma}
\begin{proof}
    If the series does not terminate then asymptotically $a_n\sim C(2\nu)^n/n!$ for some nonzero constant $C$ as $n\to\infty$.
    Therefore $f(r)\sim e^{2\nu r}$, hence $X(r)\sim e^{\nu r}$ as $r\to\infty$ which means that it is not normalisale.
\end{proof}
So necessarily there is some $N$ such that $2\nu N-\beta=0$, so $\nu=\beta/(2N)$.
We can work out the energy eigenfunction from that:
$$\nu^2=-\frac{2mE}{\hbar^2},\beta=\frac{e^2m}{2\pi\epsilon_0\hbar^2}\implies E=E_N=-\frac{e^4m}{32\pi^2\epsilon_0^2\hbar^2}\frac{1}{N^2}$$
which is exactly the energy levels deduced by Bohr's radius.
Would this mean that Bohr's model is fully correct?
Not really, as there can be solutions that are not spherically symmetric.\\
Let's find the eigenfunctions now.
Plugging the relation between $N,\nu$ and $\beta$ we have
$$\frac{a_n}{a_{n-1}}=-2\nu\frac{N-n}{n(n+1)}$$
$N=1$ gives a constant $f=f_1$, so $X_1(r)\propto e^{-\nu r}$.
$N=2$ gives $f_2(r)\propto 1-\nu r$, so $X_2(r)\propto (1-\nu r)e^{-\nu r}$.
$N=3$ gives $f_3(r)\propto 1-2\nu r+2\nu^2r^2/3$, so $X_3(r)=(1-2\nu r+2\nu^2r^2/3)e^{-\nu r}$.
And in general $f_N(r)=L_N(\nu r)$ where $L_N$ is the Legendre polynomial of order $N-1$ and hence $X_N(r)=L_N(\nu r)e^{-\nu r}$.
It is easy enough to normalise this for small $N$.
For the ground state $N=1$, the normalised wavefunction is
$$X_1(r)=\frac{\nu^{3/2}}{\sqrt{\pi}}e^{-2\nu r}=\frac{1}{\sqrt\pi}\left( \frac{e^2m}{4\pi \epsilon_0\hbar^2} \right)^{3/2}e^{-2\nu r}$$
With a little calculation we have $\langle r\rangle_{X_1}=3a_0/2$ where $a_0=2/\beta$ is the Bohr radius.
\subsection{Angular Momentum in Quantum Mechanics}
In classical dynamics, we defined the angular momentum $\underline{L}=\underline{x}\times\underline{p}$ which is a conserved quantity subject to spherically symmetrical potential.
Whereas in quantum mechanics, we have an analogous notion.
\begin{definition}
    The angular momentum operator is defined by $\underline{\hat{L}}=\underline{\hat{x}}\times\underline{\hat{p}}=-i\hbar\underline{x}\times\nabla$.
\end{definition}
Consequently components of $\underline{\hat{L}}$ are Hamiltonian and $[\hat{L}_i,\hat{L}_j]=i\hbar\epsilon_{ijk}\hat{L}_k\neq 0$ for $i\neq j$.
Therefore components of $\underline{\hat{L}}$ cannot be simultaneously measured.
\begin{definition}
    The total angular momentum operator is defined as $\hat{L}^2=|\hat{L}|^2=\hat{L}_1^2+\hat{L}_2^2+\hat{L}_3^2$.
\end{definition}
Then $[\hat{L}^2,\hat{L}_i]=0$ for all $i$ and $[\hat{H},\hat{L}^2]=0$ if $U$ is spherically symmetric.
To see the latter property, just observe that $[\hat{L}_i,\hat{x}_j]=i\hbar\epsilon_{ijk}\hat{x}_k$ and $[\hat{L}_i,\hat{p}_j]=i\hbar\epsilon_{ijk}\hat{p}_k$ and write $\hat{H}=\hat{p}^2/2m+\hat{U}$.
Consequently, fixing any $i$, $\{\hat{H},\hat{L}^2,\hat{L}_i\}$ is a set of $3$ mutually commuting operators.
Choose $i=3$, then $\hat{L}_3=\hat{L}_z$ is the $z$ component of angular momentum.
We can find simultaneous eigenstates of all three operators $\{\hat{H},\hat{L}^2,\hat{L}_3\}$.
The corresponding eigenvalues are observables.
Also, this set if maximal in the sense that there does not exists another independent operator other than $\hat{I}$ which commutes with all three.\\
What are the eigenfunctions?
We first try to find joint eigenfunctions of $\hat{L}^2,\hat{L}_3$.
In spherical polar coordinate, we can expand
$$\hat{L}^2=-\frac{\hbar^2}{\sin^2\theta}\left( \sin\theta\frac{\partial}{\partial\theta}\left( \sin\theta\frac{\partial}{\partial\theta} \right)+\frac{\partial^2}{\partial\phi^2} \right),\hat{L}_3=-i\hbar\frac{\partial}{\partial\phi}$$
If $Y=Y(\theta,\phi)$ is an eigenfunction of $\hat{L}_3$, then $\hat{L}_3Y=\hbar mY$ for some $m$.
Consequently $-iY_\phi=mY$.
Seperation of variables $Y(\theta,\phi)=y(\theta)x(\phi)$ gievs $x_\phi=imx$, so $x(\phi)=e^{im\phi}$.
But $x$ has to have period $2\pi$, so $m\in\mathbb Z$.
Now $\hat{L}^2(y(\theta)e^{im\phi})=\lambda y(\theta)e^{im\phi}$ gives
$$\frac{1}{\sin\theta}\frac{\partial}{\partial\theta}\left( \sin\theta\frac{\partial y}{\partial\theta} \right)-\frac{m^2}{\sin^2\theta}y(\theta)=-\frac{\lambda}{\hbar^2}y(\theta)$$
Which is Legendre equation.
By our study in Methods, we have
$$y_l(\theta)=P_{l,m}(\cos\theta)=(\sin\theta)^{|m|}\left.\frac{\mathrm d^{|m|}}{\mathrm dx^{|m|}}P_l(x)\right|_{x=\cos\theta}$$
where $P_l$ are the ordinary Legendre polynomials and $\lambda=l(l+1)\hbar^2$ for $-l\le m\le l$.
So the simultaneous eigenfunctions of $\hat{L}^2$ and $\hat{L}_3$ are labelled by $l=0,1,2,\ldots,m\in\{-l,\ldots,l\}$ and take the form
$$y_{l,m}(\theta,\phi)=P_{l,m}(\cos\theta)e^{im\phi}$$
with eigenvalues $m\hbar$ and $l(l+1)\hbar^2$ respectively.
Physically, we interpret $l$ as the total angular momentum numbers and $m$ the azimuthal quantum numbers.
Note that the constraint $m\in[-l,l]$ corresponds to the classical mechanical result of $-|L|\le L_3\le |L|$.\\
We can compute (whyyyyyyyy) some of these eigenfunctions (known as spherical harmonics):
$$Y_{0,0}=\frac{1}{\sqrt{4\pi}},Y_{1,0}=\sqrt{\frac{3}{4\pi}}\cos\theta,Y_{1,\pm 1}(\theta,\phi)=\mp\sqrt{\frac{3}{8\pi}}\sin\theta e^{\pm i\phi}$$
$$Y_{2,0}(\theta,\phi)=\sqrt{\frac{5}{16\pi}}(3\cos^2\theta-1),Y_{2,\pm 1}(\theta,\phi)=\mp\sqrt{\frac{15}{8\pi}}\sin\theta\cos\theta e^{\pm i\phi}$$
$$Y_{2,\pm 2}(\theta,\phi)=\sqrt{\frac{15}{32\pi}}\sin^2\theta e^{\pm 2i\phi}$$
\subsection{Full Wavefunction of the Hydrogen Atom}
Recall that in spherical polar,
$$\nabla^2X=\frac{1}{r}\frac{\partial^2(rX)}{\partial r^2}+\frac{1}{r^2\sin^2\theta}\left( \sin\theta\frac{\partial}{\partial\theta}\left( \sin\theta\frac{\partial X}{\partial\theta} \right)+\frac{\partial^2X}{\partial\phi^2} \right)$$
and
$$\hat{L}^2X=\frac{\hbar^2}{\sin^2\theta}\left( \sin\theta\frac{\partial}{\partial\theta}\left( \sin\theta\frac{\partial X}{\partial\theta} \right) +\frac{\partial^2}{\partial\phi^2}\right)$$
Therefore
$$-\hbar^2\nabla^2X=-\frac{\hbar^2}{r}\frac{\partial^2(rX)}{\partial r^2}+\frac{\hat{L}^2X}{r^2}$$
which means that we can rewrite the Hamiltonian as
$$\hat{H}X=-\frac{\hbar^2}{2m}\left( \frac{\partial^2X}{\partial r^2}+\frac{2}{r}\frac{\partial X}{\partial r} \right)+\frac{\hat{L}^2X}{2mr^2}-\frac{e^2}{4\pi\epsilon_0r}X$$
As $\{\hat{H},\hat{L}^2,\hat{L}_3\}$ is a set of commuting operators, we look for solutions to $\hat{H}X=EX$ of the form $X(r,\theta,\phi)=g(r)Y_{l,m}(\theta,\phi)$, which (after a lot of calculations) yields
$$\frac{\mathrm d^2g}{\mathrm dr^2}+\frac{2}{r}\frac{\mathrm dg}{\mathrm dr}-\frac{l(l+1)}{r^2}g+\frac{\beta}{r}g-\nu^2g=0,\nu=\sqrt{-\frac{2mE}{\hbar^2}},\beta=\frac{e^2m}{2\pi\epsilon_0\hbar^2}$$
The rest is similar to what we did for the spherically symmetrical case.
We seek solutions of the form $g(r)=f(r)e^{-\nu r}$ due to its limiting behavour, which gives
$$\frac{\mathrm d^2f}{\mathrm dr^2}+\frac{2}{r}(1-\nu r)\frac{\mathrm df}{\mathrm dr}-\frac{l(l+1)}{r^2}f+\frac{1}{r}(\beta-2\nu)f=0$$
The same procedure of finding series solution applies.
Since $0$ is a regular singular point, we try solutions of the form $f(r)=r^\sigma\sum_na_nr^n$.
By routine work of substituting we get the indicial equation $\sigma(\sigma-1)+2\sigma-l(l+1)=0$, hence $\sigma=l$ or $\sigma=-l-1$.
The latter case has to be discarded as we don't want singularity at $0$.
Therefore $f(r)=r^l\sum_na_nr^n$.
Again routine substitution work yields the recurrence
$$a_n=\frac{2\nu(n+l)-\beta}{n(n+2l+1)}a_{n-1}$$
which determines the series completely up to a constant.
The same trick as before applies:
If the series does not terminate, then by estimating the order of growth of $(a_n)$ we find that the $g$ that we want will not be normalisable.
Therefore the series must terminate.
This means that (for nonzero $a_0$) there exists some $n_{\max{}}$ such that $a_{n_{\max{}}}=0$ but $a_{n_{\max{}}-1}\neq0$ which has to satisfy
$$2\nu(n_{\max{}}+l)-\beta=0\implies \nu=\frac{\beta}{2N},N=n_{\max{}}+l$$
by plugging in the respective definition, the energy levels are
$$E=E_N=-\frac{e^4m}{32\pi^2\epsilon_0^2\hbar^2}\frac{1}{N^2}$$
So the spectrum of energy levels is identical to what we obtained in the special case where the eigenfunctions are spherically symmetric.
But in this more general setting, the eigenvalues can have some other (non-spherically-symmetric) eigenfunctions.
How many of them?
It's not hard to see:
For each $N$, there are exactly
$$D(N)=\sum_{l=0}^{N-1}\sum_{n=-l}^l1=\sum_{l=0}^{N-1}(2l+1)=N^2$$
many different eigenfunctions.
This is called the degeneracy of the energy value $E_N$.
So the full spectrum of the eigenfunctions of the Hydrogen atom is then
$$X_{N,l,m}(r,\theta,\phi)=\xi^lL_{N,l}(\xi)e^{-\xi}Y_{l,m}(\theta,\phi),\xi=\frac{\beta r}{2N}=\frac{e^2mr}{4N\pi\epsilon_0\hbar^2}$$
where $L_{N,l}$ (whose coefficients are defined by our terminating series earlier) are called the generalised Laguerre polynomials and $Y_{l,m}$ are the spherical harmonics.\\
$N=1,2,3,\ldots$ are called the principal quantum numbers, $l=0,\ldots,N-1$ are the total angular momentum quantum numbers and $m\in\{-l,\ldots,l\}$ are the azimuthal quantum numbers.
The Bohr model captured the case where $m=l\simeq N>>1$.
In this case $L_3==m\hbar\simeq N\hbar$ and $L=\sqrt{L^2}=\sqrt{l(l+1)\hbar^2}\simeq N\hbar$.\\
Note that due to the definition of the spherical harmonics, we have
$$\int_0^{2\pi}\int_{-1}^1\int_0^\infty |X_{N,l,m}(r,\theta,\phi)|^2r^2\,\mathrm dr\mathrm d(\cos\theta)\mathrm d\phi=\int_0^\infty |g(r)|^2r^2\,\mathrm dr$$
So we define $P(r)|g(r)|^2r^2$ as the radial probability.
Asymptotically,
$$P(r)=r^2|g(r)|^2\sim r^{2(l+1)}\exp\left( -\frac{\beta r}{2(l+1)} \right)$$
which, in the Bohr limit, just gives $r^{2N}\exp(-\beta r/N)$.
By differentiating, the stationary points of $P$ (i.e. the modes) are approximately at $r_{\rm peak}\simeq 2N^2/\beta=N^2r_1$ which is the same as what we obtained using Bohr's model.
\subsection{Towards Periodic Table}
If we have a nucleus with charge $+ze$ and there are $z$ electrons around it which are treated as independent particles (so ignore any interactions betweem electrons like electromagnetism and gravity), then the eigenfunction can be seperated as $X(\underline{x}_1,\ldots,\underline{x}_z)=X(\underline{x}_1)\cdots X(\underline{x}_z)$ where each $X(\underline{x}_i)$ is a rescaled solution for the Hydrogen atom.
Its energy level is then the sum of energies of all electrons $E=\sum_iE_i$.
which works for small $z$, but the interaction force between electrons is growing when $z$ is large -- which is why we need to develop more theory in next year's Principle of Quantum Mechanics.
\end{document}