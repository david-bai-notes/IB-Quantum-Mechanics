\section{Solutions of TISE in One Dimension}
In one dimension, TISE restricts to
$$-\frac{\hbar^2}{2m}X^{\prime\prime}+UX=EX$$
where $E\in\mathbb R$.
\subsection{Infinte Potential Well}
Consider the potential
\footnote{Unorthodox, I know. I have stopped caring.}
$$U(x)=\begin{cases}
    0\text{, if $|x|\le a$}\\
    \infty\text{, if $|x|>a$}
\end{cases}$$
For $|x|>a$ we take the solution $X(x)=0$.
And we also want $X(\pm a)=0$ as we want $X$ to be continuous.
For $|x|\le a$, we are then aiming at the boundary value problem
$$-\frac{\hbar^2}{2m}X^{\prime\prime}=EX\iff X^{\prime\prime}+K^2X=0,K^2=\frac{2mE}{\hbar^2}\ge 0$$
subject to $X(\pm a)=0$.
This is known to have the general solution $X(x)=A\sin(Kx)+B\cos(Kx)$ for constants $A,B$.
The boundary conditions then require either $A=0$ and $K=n\pi/2a$ for $n=1,3,5,\ldots$ or $B=0$ and $K=n\pi/2a$ with $n=2,4,6,\ldots$.
So the allowed values of the energy would be
$$E_n=\frac{\hbar^2\pi^2}{8ma^2}n^2,n=1,2,3,\ldots$$
The lowest positive energy (aka ground state energy) is then $E_1=\hbar^2\pi^2/(8ma^2)$.
The solutions are
$$X_n(x)=\frac{1}{\sqrt{a}}\begin{cases}
    \cos(n\pi x/(2a))\text{, for $n=1,3,5,\ldots$}\\
    \sin(n\pi x/(2a))\text{, for $n=2,4,6,\ldots$}
\end{cases}$$
where the factor is obtained from the assumption that $X_n$ is normalised.
I am too lazy to plot the functions, but whoever read this are encouraged to plot a few of these states.
Without plotting, however, one can immediately realise that when $n$ is large, $X$ tends to fluctuate a lot.
The solution also allows us to draw the analogy between these solutions and standing waves with two endpoints fixed at $\pm a$.
Also, $X_n(-x)=(-1)^{n+1}X_n(x)$, so $X_n$ is either even or odd.
This is actually a general feature of the eigenfunctions of the Hamiltonian $\hat{H}$.
\begin{proposition}
    If $U$ is even and the energy spectrum is non-degenerate (i.e. we require $E_1<E_2<\cdots$), then each eigenfunction is either even or odd.
\end{proposition}
\begin{proof}
    If $U$ is even then TISE is reflection invariant, so $x\mapsto X(-x)$ is also a solution.
    But it also has energy $E$, so $X(-x)=AX(x)$ where $A$ is a constant.
    But then $X(x)=X(-(-x))=AX(-x)=A^2X(x)$, so $A=\pm 1$ as $X$ is not identically zero.
\end{proof}
\subsection{Finite Potential Well}
We consider the potential of the form
$$U(x)=\begin{cases}
    0\text{, if $|x|\le a$}\\
    U_0\text{, if $|x|>a$}
\end{cases}$$
where $U_0>0$.
We only consider the case $E<U_0$.
The other case will be dealt with later.
Also we only look for even $X$.
We define the nonnegative real quantities
$$K=\sqrt{\frac{2mE}{\hbar^2}},\kappa=\sqrt{\frac{2m(U_0-E)}{\hbar^2}}$$
For $|x|\le a$, we obtain the general solution $X(x)=A\sin(Kx)+B\cos(Kx)$ for $A,B$ constants.
We are looking for even states only, so we might as well write $X(x)=A\cos(Kx)$.
For $|x|>a$, we have $X^{\prime\prime}-\kappa^2X=0$ which has solution $X=Ce^{\kappa x}+De^{-\kappa x}$ for some constants $C,D$.
But the normalisation condition then tells us $C=0$ for $x>a$ and $D=0$ for $x<-a$.
Hence
$$X(x)=\begin{cases}
    Ce^{\kappa x}\text{, for $x<-a$}\\
    A\cos(Kx)\text{, for $|x|\le a$}\\
    De^{-\kappa x}\text{, for $x>a$}
\end{cases}$$
The continuity of $X$ and $X^\prime$ that we assumed then yield $K\tan(Ka)=\kappa$.
But the definition of $K$ and $\kappa$ yields $K\tan(Ka)=\kappa$.
We also know that $K^2+\kappa^2=2mU_0/\hbar^2$.
Define $\xi=Ka$ and $\eta=\kappa a$, then the two equations become
$$\begin{cases}
    \xi\tan\xi=\eta\\
    \eta^2+\eta^2=r_0^2
\end{cases}$$
where $r_0^2=2a^2mU_0/\hbar^2$.
By a plot, we know that there are only finitely many solutions $\{\xi_1,\ldots,\xi_p\}$ of $\xi$ with $(n-1)\pi\le\xi_n\le(n-1/2)\pi$ and the allowed energy are of the form
$$E_n=\frac{\hbar^2}{2ma^2}\xi_n^2,n=1,\ldots,r$$
As $U_0\to\infty$, we can get $r_0\to\infty$, so $p\to\infty$.
And the allowed energy would be
$$E_n=\frac{\hbar^2}{2ma^2}\xi_n^2=\frac{\hbar^2(2n-1)^2}{2ma^2}\pi^2$$
which is consistent with our result on infinite well.
Of course we can determine all the eigenfunctions as we have enough number of conditions on the breaking points and the normalisation condition.
One can also show by some calculation that the eigenfunctions also go to the eigenfunctions of the infinite well solution as $U_0\to\infty$.
\subsection{Free Particle}
If we require that
$$\int_{-\infty}^\infty |X(x)|^2\,\mathrm dx=N\in\mathbb R_+$$
then necessarily
$$\lim_{R\to\infty}\int_{|x|>R}|X(x)|^2\,\mathrm dx=0$$
Returning to our TISE $X^{\prime\prime}+K^2X(x)=0$ where $K=\sqrt{2mE/\hbar^2}$ which has solutions in the form $X_K(x)=Ae^{iKx}$.
This is obviously a continuous spectrum of solutions (corresponding to different values of $E$).
They gives the solutions $\psi_K(x,t)=X_k(x)e^{-i\hbar K^2t/(2m)}$.
But they are not in general normalisable!
Indeed,
$$\int_{\mathbb R}|\psi_K(x,t)|^2\,\mathrm dx=\int_{\mathbb R}|X_K(x,t)|^2\,\mathrm dx=|A|^2\int_{\mathbb R}\mathrm dx=\infty$$
unless $A=0$.
The workaround is to consider the integral superposition
$$\psi(x,t)=\int_{\mathbb R}A(K)\psi_K(x,t)\,\mathrm dK=\int_{\mathbb R}A(K)e^{iKx}e^{-i\hbar K^2t/(2m)}\,\mathrm dK$$
and choose $A$ that decreases fast enough as $K\to\infty$ so that $\psi$ is well-defined and normalisable.
A typical choice is the Gaussian wavepackage
$$A(K)=\exp\left( -\frac{\sigma}{2}(K-K_0)^2 \right)$$
where $\sigma>0$.
So using this we can write
$$\psi(x,t)=\int_{\mathbb R}\exp F(K)\,\mathrm dK$$
where
\begin{align*}
    F(K)&=-\frac{\sigma}{2}(K-K_0)^2+i\left(Kx-\frac{\hbar K^2}{2m}t\right)\\
    &=-\frac{1}{2}\left( \sigma+\frac{i\hbar t}{m} \right)K^2+(K_0\sigma+ix)K-\frac{\sigma}{2}K_0^2\\
    &=-\frac{\alpha}{2}\left( K-\frac{\beta}{\alpha} \right)^2+\frac{\beta^2}{2\alpha}+\delta,\alpha=\sigma+\frac{i\hbar t}{m},\beta=K_0\sigma+ix,\delta=-\frac{\sigma}{2}K_0^2
\end{align*}
Therefore,
\begin{align*}
    \psi(x,t)(x,t)&=\exp\left( \frac{\beta^2}{2\alpha}+\delta \right)\int_{-\infty}^\infty\exp\left( -\frac{\alpha}{2}\left( K-\frac{\beta}{\alpha} \right)^2 \right)\,\mathrm dK\\
    &=\exp\left( \frac{\beta^2}{2\alpha}+\delta \right)\int_{-\infty-i\nu}^{\infty-i\nu}\exp\left( -\frac{\alpha}{2}\tilde{K}^2 \right)\,\mathrm d\tilde{K},\tilde{K}=K-\frac{\beta}{\alpha},\nu=\operatorname{Im}\frac{\beta}{\alpha}\\
    &=\sqrt{\frac{2\pi}{\alpha}}\exp\left( \frac{\beta^2}{2\alpha}+\delta \right)
\end{align*}
I know the last step is staggering.
It's ok, you are stong enough to handle these traumas in your life.
Anyways, putting everything back gives
$$\psi(x,t)=\sqrt{\frac{2\pi}{\alpha}}\exp\left( -\frac{\sigma(x-\hbar K_0t/m)^2}{2(\sigma^2+\hbar^2t^2/m^2)} \right)$$
And (with the help of God) we can obtain the probability density
$$\rho(x,t)=\frac{C}{\sqrt{\sigma^2+\hbar^2t^2/m^2}}\exp\left( -\frac{\sigma(x-\hbar K_0t/m)^2}{\sigma^2+\hbar^2t^2/m^2} \right)$$
where $C$ is a normalisation constant, which can be determined by more calculation:
$$\int_{-\infty}^\infty\rho(x,t)\,\mathrm dx=1\implies C=\sqrt{\frac{\sigma}{\pi}}$$
Now, by a little calculation we obtain the mean $\langle x\rangle=\hbar K_0t/m$.
We write $v=\hbar K_0/m$ as an analogy of velocity.
The standard deviation is
$$\Delta x=\sqrt{\langle x^2\rangle-\langle x\rangle^2}=\sqrt{\frac{1}{2}\left( \sigma+\frac{\hbar^2t^2}{m^2\sigma} \right)}$$
which increases as $t\to\infty$, therefore the distribution is more spreaded out as $t\to\infty$.\\
Back to $\psi_K(x,t)=Ae^{iKx}e^{-i\hbar K^2t/(2m)}$, which as one can verify has standard deviation $\Delta x=\infty$ on position and $\Delta p=0$ on momentum since the momentum is just $p=\hbar K$.
These whole $\Delta x,\Delta p$ business will be dealt with later when we get to Heisenberg's Uncertainty Principle.
In fact, it is possible to show that the Gaussian wavepackage minimises uncertainty.
Now $\rho_K(x,t)=|A(K)|^2$ is then interpreted as the constant average density of a particle.
The probability current is
$$J_K(x,t)=-\frac{i\hbar}{2m}\left( \psi_K^*\frac{\partial\psi_K}{\partial x}-\psi_K\frac{\partial\psi_K^*}{\partial x} \right)=|A|^2\frac{\hbar K}{m}=|A|^2\frac{p}{m}$$
This is basically the product of the average density and velocity, which can be interpreted as the average flux of particles.
In some way, we are interpreting the state as a beam of particles where $A$ relates to the beam density.
\subsection{Scattering States}
Consider $U_0>0$ and
$$U(x)=\begin{cases}
    U_0\text{, for $x\in (0,a)$}\\
    0\text{, otherwise}
\end{cases}$$
Define
$$R=\lim_{t\to\infty}\int_{-\infty}^0|\psi(x,t)|^2\,\mathrm dx,T=\lim_{t\to\infty}\int_0^\infty|\psi(x,t)^2|\,\mathrm dx$$
If we interpret the system as a wave moving towards $x=0$ being scattered (partially reflected and partially transmitted), then $R$ can be seens as the reflection probability and $T$ the transmission probability.
If $\psi$ is normalised, then we have $R+T=1$.
Take the solutions $\psi_K(x,t)=X_K(x)e^{-i\hbar K^2t/(2m)}$ where $X_K(x)=Ae^{iKx}$.
\subsubsection{Scattering off a Potential Step}
Take $a\to\infty$ gives the potential
$$U(x)=\begin{cases}
    0\text{, for $x\le 0$}\\
    U_0\text{, for $x>0$}
\end{cases}$$
Restate the TISE
$$-\frac{\hbar^2}{2m}X^{\prime\prime}+UX=EX$$
First, we consider the case $E>U_0$.
For $x\le 0$, if we throw in $K=\sqrt{2mE/\hbar^2}$ the system gives
$$X_K(x)=X_K^{(+)}(x)+X_K^{(-)}(x),X_K^{(+)}(x)=Ae^{iKx},X_K^{(-)}(x)=Be^{-iKx}$$
where $A,B$ are constants.
Then we interpret $X_K^{(+)}$ as the beam of incident particles from $x=-\infty$ with $p=\hbar K$ and probability current $J_K^{(+)}=|A|^2\hbar K/m$.
The $X_K^{(-)}$ is correspondingly interpreted as the beam of reflected particles going towards $x=-\infty$ with $p=-\hbar k$ and $J_K^{(-)}=-|B|^2\hbar K/m$.
Now for $x>0$ we have $X^{\prime\prime}+\tilde{K}^2X=0$ where $\tilde{K}=\sqrt{2m(E-U_0)/\hbar^2}$ which is well-defined as we assumed $U_0<E$.
This yield the solutions
$$X_{\tilde{K}}=X_{\tilde{K}}^{(+)}+X_{\tilde{K}}^{(+)},X_{\tilde{K}}^{(+)}(x)=Ce^{i\tilde{K}x},X_{\tilde{K}}^{(-)}(x)=De^{-i\tilde{K}x}$$
Again $X_{\tilde{K}}^{(+)}$ is the beam of transmitted particles moving towards $x=+\infty$ with $p=\hbar\tilde{K}$ and $X_{\tilde{K}}^{(-)}$ is the beam if incident particles from $x=+\infty$ with $p=-\hbar \tilde{K}$.
We only want to consider the scattering problem, so we can set $D=0$.
Continuity of $X$ at $x=0$ gives $A+B=C$ and continuity of $X^\prime$ at $x=0$ gives $iKA-iKB=i\tilde{K}C$, therefore
$$B=\frac{K-\tilde{K}}{K+\tilde{K}}A,C=\frac{2K}{K+\tilde{K}}A$$
So the incident, reflective and transmitted probability current are
$$J_{\rm inc}=\frac{\hbar K}{m}|A|^2,J_{\rm ref}=\frac{\hbar K}{m}\left( \frac{K-\tilde{K}}{K+\tilde{K}} \right)^2|A|^2,J_{\rm tr}=\frac{\hbar \tilde{K}}{m}\frac{4K^2}{(K+\tilde{K})^2}|A|^2$$
Hence
$$R=\frac{J_{\rm ref}}{J_{\rm inc}}=\left( \frac{K-\tilde{K}}{K+\tilde{K}} \right)^2,T=\frac{J_{\rm tr}}{J_{\rm inc}}=\frac{4K\tilde{K}}{(K+\tilde{K})^2}$$
Easy to verify that $R+T=1$ and when $E\to \infty$ we have $K-\tilde{K}\to 0$ which implies $R\to 0,T\to 1$, which corresponds to the classical case.\\
For the case $E<U_0$, the equations become
$$\begin{cases}
    X^{\prime\prime}+K^2X=0\implies X=Ae^{iKx}+Be^{-iKx}\text{, for $x\le 0$}\\
    X^{\prime\prime}-\kappa^2X=0\implies X=Ge^{\kappa x}+Fe^{-\kappa x}\text{, for $x>0$}
\end{cases}$$
where as you expect,
$$K=\sqrt{\frac{2mE}{\hbar^2}},\kappa=\sqrt{\frac{2m(U_0-E)}{\hbar^2}}$$
We have $G=0$ because we expect the wavefunction to not blow up at $+\infty$.
Hence
$$X(x)=\begin{cases}
    Ae^{iKx}+B^{-iKx}\text{, for $x\le 0$}\\
    Fe^{-\kappa x}\text{, for $x>0$}
\end{cases}$$
Then the assumed continuity of $X,X^\prime$ at $0$ gives
$$B=\frac{iK+\kappa}{iK-\kappa}A,F=\frac{2iK}{iK-\kappa}A$$
So
$$j_{\rm tr}(x)=0,j_{\rm inc}=\frac{\hbar K}{m}|A|^2,j_{\rm ref}=\frac{\hbar K}{m}|B|^2=\frac{\hbar K}{m}|A|^2$$
Hence $R=1$ and $T=0$ just as we expect it.
\subsubsection{Scattering off a Potential Barrier}
Consider $a\in (0,\infty)$, so
$$U(x)=\begin{cases}
    U_0\text{, for $x\in (0,a)$}\\
    0\text{, otherwise}
\end{cases}$$
We deal with the $E\le U_0$ case first.
We define
$$K=\sqrt{\frac{2mE}{\hbar^2}},\bar{K}=\sqrt{\frac{2m(U_0-E)}{\hbar^2}}$$
Therefore
$$\begin{cases}
    X^{\prime\prime}(x)-\bar{K}^2X(x)=0\text{, for $x\in (0,a)$}\\
    X^{\prime\prime}(x)+K^2X(x)=0\text{, otherwise}
\end{cases}$$
which gives the general (normalised) solution
$$X(x)=\begin{cases}
    e^{iKx}+Ae^{-iKx}\text{, for $x\le 0$}\\
    Be^{-\bar{K}x}+Ce^{\bar{K}x}\text{, for $x\in (0,a)$}\\
    De^{iKx}+Fe^{-iKx}\text{, for $x\ge a$}
\end{cases}$$
We can set $F=0$ since we only consider the case where the incident beam is from $x=-\infty$.
By continuity of $X,X^\prime$ again we get
$$D=\frac{-4i\bar{K}K}{(\bar{K}-iK)^2e^{(\bar{K}+iK)a}-(\bar{K}+iK)^2e^{-(\bar{K}-iK)a}}$$
In classical dynamics, the particle can never pass the barrier.
But our result just shows that this is not in general the case, since
$$T=\frac{j_{\rm tr}}{j_{\rm inc}}=\frac{\hbar K|D|^2/m}{\hbar K/m}=|D|^2=\frac{4K^2\bar{K}^2}{(K^2+\bar{K}^2)^2\sinh^2(\bar{K}a)+4K^2\bar{K}^2}\neq 0$$
In particular, for $\bar{K}a>>1$, either $a>>1$ or $U_0>>E$ and
$$T\sim\frac{16K^2\bar{K}^2}{(K^2+\bar{K}^2)^2}e^{-2\bar{K}a}$$
So if the barrier is very wide or very tall, $T$ goes exponentially to $0$.
When $E<U_0$ but a beam of particles is transmitted, this phenomenon is called quantum tunneling.
\subsubsection{Physical Examples of Quantum Tunneling}
\begin{example}[Cold Emission]
    When light irradiates a metal surface, electrons are emitted when $\hbar\omega>W$ where $\omega$ is the angular frequency of photons and $W$ is a constant (called the work function) associated with the metal.
    An electron would be facing some sort of triangular barrier.
    In addition to the potential step at $x=a$ because of the work function $W$, suppose $\xi$ is the external electric field, then the resulting potential induced by the Lorentz force is $V_{\rm Lorentz}(x)=-e\xi(x-a)$.
    Their superposition would be a triangular barrier with cliff at $x=a$.
    So even for an electron with energy less than $W$, it can go to the part of the potential where it is low enough via quantum tunneling and excite from there.
    This is knowns as cold emission in photoelectric effect.
\end{example}
\begin{example}[Radioactive Decay]
    Consider the decay $\text{N}_{Z}^A\to \text{M}_{Z-2}^{A-4}+\text{He}_2^4$.
    For the $\alpha$ particle $\text{He}_2^4$, we want to know the potential field it is in.
    Indeed, if we let $r$ to be the distant between this $\alpha$ particle and the center of the nucleus, then $U(r)$ will begin negative near $0$ because of the attractive nuclear force and eventually go up positive again after $r$ gets large enough since the repulsive Coulumb force comes in play.
    Eventually, it goes down (but keep positive) and tends to $0$ as $r\to\infty$.
    This is known as the Gamow model.
    The curve certainly looks like a potential barrier.
    If we let $T$ be the transmission coefficient, then the half-life of this decay is actually proportional to $T^{-1}$.
\end{example}
\subsection{The Harmonic Oscillator}
The potential of the harmonic oscillator can be written as $U(x)=Kx^2/2=m\omega^2x^2/2$ where $K>0, \omega=\sqrt{k/m}>0$.
In classical mechanics, Newton's Second Law gives the equation $\ddot{x}=-\omega^2x$ which has solutions $x=A\sin(\omega t)+B\cos(\omega t)$ where $A,B$ are constants.
So the particle oscillates about $x=0$ with period $2\pi/\omega$.\\
In quantum mechanics, this gives the TISE
$$-\frac{\hbar^2}{2m}\frac{\mathrm d^2X}{\mathrm dx^2}+\frac{1}{2}m\omega^2x^2X(x)=EX(x)$$
where we expect to find a discrete set of normalisable eigenfunctions by looking at the shape of the potential.
As seen earlier, we can guarantee to find eigenfunctions that are either even or odd.
Write $\xi^2=m\omega x^2/\hbar$ and $\mathcal E=2E/(\hbar\omega)$, then this change of variables yield
$$-\frac{\mathrm d^2X}{\mathrm d\xi^2}+\xi^2X(\xi)=\mathcal EX(\xi)$$
For $\mathcal E=1$ it reduces to $-X^{\prime\prime}(\xi)+\xi^2X=X$.
Educated guess reveals that $X_0=A\exp(-\xi^2/2)$ is a family of solutions which is normalisable when $A\neq 0$.
This gives one pair of eigenvalue $E_0=\hbar\omega/2$ and eigenfunction $X_0(x)=A\exp(-m\omega x^2/(2\hbar))$ for $A\neq 0$.\\
For the general case, set $X(\xi)=f(\xi)\exp(-\xi^2/2)$, which transforms the equation into
$$-\frac{\mathrm d^2f}{\mathrm d\xi^2}+2\xi\frac{\mathrm df}{\mathrm d\xi}+(1-\mathcal E)f=0$$
Now we are desperate, so the obvious thing to do is to plug in the power series solution $f(\xi)=\sum_na_n\xi^n$ which gives
$$0=\sum_{n=0}^\infty [(n+1)(n+2)a_{n+2}-2na_n+(\mathcal E-1)a_n]\xi^n$$
which gives the recurrence
$$a_{n+2}=\frac{2n-\mathcal E+1}{(n+1)(n+2)}a_n$$
The recurrence steps by $2$, so it has even or odd solutions.
Here we need to prove an important claim
\begin{claim}
    Suppose the series does not terminate then $X(\xi)=f(\xi)\exp(-\xi^2/2)$ is not normalisable.
\end{claim}
\begin{proof}
    $a_{n+2}/a_n\sim 2/n$ as $n\to\infty$, which means that $f(\xi)\sim\exp(\xi^2)$, hence $X$ is not normalisable.
\end{proof}
Therefore, for a normalisable solution to exist, $\mathcal E$ must be an odd positive integer, so we get the eigenvalues and eigenfunctions
$$E_N=\left( N+\frac{1}{2} \right)\hbar\omega,X_N(x)=f_N\left( \sqrt{\frac{m\omega}{\hbar}} \right)\exp\left( -\frac{m\omega x^2}{2\hbar} \right)$$
where $f_N$ are the respective polynomials that yield from the series solution as the choice of $\mathcal E$ makes it terminate.
These are called Hermite polynomials which, as one can verify, satisfies $f_N(-\xi)=(-1)^Nf_N(\xi)$ and
$$f_N(\xi)=(-1)^n\exp(\xi^2)\frac{\mathrm d^n}{\mathrm d\xi^n}\exp(-\xi^2)$$
We've got nothing better to do so why not calculate some of them.
\begin{center}
    \begin{tabular}{c|c|c|c}
        $N$&$f_N(\xi)$&$E_N$&$X_N(\xi)$\\ \hline
        $0$&$1$&$\hbar\omega/2$&$\exp(-\xi^2/2)$\\
        $1$&$\xi$&$3\hbar\omega/2$&$\xi\exp(-\xi^2/2)$\\
        $2$&$1-2\xi^2$&$5\hbar\omega/2$&$(1-2\xi^2)\exp(-\xi^2/2)$\\
        $3$&$\xi-2\xi^3/3$&$7\hbar\omega/2$&$(\xi-2\xi^3/3)\exp(-\xi^2/2)$
    \end{tabular}
\end{center}