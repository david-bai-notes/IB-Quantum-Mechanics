\section{Foundation of Quantum Mechanics}
\subsection{Quantum Mechanics of a Particle}
In classical dynamics, a particle is characterised by its position $\underline{x}$ and momentum $\underline{p}$.
We formulate classical laws in this framework.
E.g. in Newton's Second Law, we have $F(\underline{x})=m\underline{\ddot{x}}$.
A simple consequence of this is that the position and momentum of a particle at $t=0$ determines its motion under physical laws formulated in this way.\\
But in Quantum Mechanics, things are different.
\begin{postulate}[State and Wavefunction]
    A particle in space is described by its state $\psi:\mathbb R^3\times\mathbb R\to\mathbb C$ such that
    $$\int_{\mathbb R^3}|\psi(\underline{x},t)|^2\,\mathrm dV=N\in\mathbb R_+$$
    The probability amplitude of finding a particle at a place $\underline{x}$ and a given time $t$ is given by the state as a wavefunction $\psi(\underline{x},t)$.
\end{postulate}
There are subtle differences when we talk about state and when we talk about a wavefunction.
But we don't care.
\begin{postulate}[Born's Rule]
    $|\bar\psi(\underline{x},t)|^2\,\mathrm dV$ is the probability of finding the particle in $\mathrm dV$.
    Here, $\bar\psi$ is the normalised wavefunction $\bar\psi=\psi/\sqrt{N}$.
\end{postulate}
\begin{postulate}[Time-Dependent Schr\"odinger's Equation (TDSE)]
    For a particle of mass $m$ in potential $U(\underline{x})$, in the time evolution of $\psi$ we have
    $$i\hbar\frac{\partial\psi}{\partial t}=-\frac{\hbar^2}{2m}\nabla^2\psi+U\psi$$
\end{postulate}
Observe that the equation is first order in $t$ and second order in $\underline{x}$.
It is also linear, so it still holds if we replace $\psi$ by a scalar multiple of it, for example $\bar\psi$.
Heuristically, one can derive this equation (in dimension one) in the following way:\\
Consider the particle as a de Broglie wave with wavefunction $e^{i(kx-\omega t)}$.
So for a free particle with $U(x)=0$, we have $E=p^2/2m$, therefore the wavefunction is
$$\exp(i(kx-\omega t))=\exp\left( \frac{1}{\hbar}(px-Et) \right)=\exp\left( \frac{1}{\hbar}\left(px-\frac{p^2}{2m}t\right) \right)$$
which can be verified a solution to the equation.
But this is just heuristic, the de Broglie wave is not even normalisable.
We will (hopefully) see the actual stuff later.\\
One thing that can easily go wrong with these postulate is that we do not know if the normalised wavefunction will continue to be normalised as time goes, so we need some work on that.
It suffices show that TDSE guarantees that $N$ does not depend on $t$.
\begin{proposition}
    $N$ does not depend on $t$ assuming TDSE.
\end{proposition}
The proof involves the use of some techniques that need some analytical justification.
But this is an applied course, so nobody cares.
\begin{proof}
    Just differentiate
    \begin{align*}
        \frac{\mathrm dN}{\mathrm dt}&=\frac{\mathrm d}{\mathrm dt}\int_{\mathbb R^3}|\psi(\underline{x},t)|^2\,\mathrm dV\\
        &=\int_{\mathbb R^3}\frac{\partial}{\partial t}\left( \psi^*(\underline{x},t)\psi(\underline{x},t) \right)\,\mathrm dV\\
        &=\int_{\mathbb R^3}\psi^*\frac{\partial\psi}{\partial t}+\psi\frac{\partial \psi^*}{\partial t}\,\mathrm dV\\
        &=\int_{\mathbb R^3}\psi^*\left( \frac{i\hbar}{2m}\nabla^2\psi+\frac{i}{\hbar}U\psi\right)+\psi\left( -\frac{i\hbar}{2m}\nabla^2\psi^*-\frac{i}{\hbar}U\psi^* \right)\,\mathrm dV\\
        &=\frac{i\hbar}{2m}\int_{\mathbb R^3}\psi^*\nabla^2\psi-\psi\nabla^2\psi^*\,\mathrm dV\\
        &=\frac{i\hbar}{2m}\int_{\mathbb R^3}\nabla\cdot( \psi^*\nabla\psi-\psi\nabla\psi^*)\,\mathrm dV\\
        &=\frac{i\hbar}{2m}\int_{\partial V}(\psi^*\nabla\psi-\psi\nabla\psi^*)\cdot\mathrm d\underline{S}\\
        &=0
    \end{align*}
    by our boundary condition.
\end{proof}
\begin{remark}
    Assume that $\psi$ is normalised.
    Write $\rho(\underline{x},t)=|\psi(\underline{x},t)|^2$ as the probability density and
    $$\underline{J}=-\frac{i\hbar}{2m}(\psi^*\nabla\psi-\psi\nabla\psi^*)$$
    the probability current.
    Then we have the conservation law
    $$\frac{\partial\rho}{\partial t}+\nabla\cdot\underline{J}=0$$
    by the calculation involved in the above proof.
\end{remark}
\subsection{Principle of Superpositions}
As TDSE is linear in $t$, for states $\phi_1,\phi_2$ satisfying it, $a_1\phi_1+a_2\phi_2$ also satisfies TDSE for any $a_1,a_2\in\mathbb C$.
Obviously, we want to show that this superposition is either $0$ or also normalisable.
\begin{proposition}
    If $\phi_1,\phi_2$ are normalisable, so is $a_1\phi_1+a_2\phi_2$ for any $a_1,a_2\in\mathbb C$ given that it is nonzero.
\end{proposition}
\begin{proof}
    Quite obvious.
\end{proof}
\begin{corollary}
    The set of all states, together with the zero function, is a vector space.
\end{corollary}
This vector space is often denoted by $\mathcal H$.
\begin{proof}
    Immediate.
\end{proof}
We can equip a complex inner product
$$(\psi,\phi)=\int_{\mathbb R^3}\psi^*\phi\,\mathrm dV$$
on this space.
If we also know that it is complete with respect to that inner product, then $\mathcal H$ is a (complex) Hilbert space.
\subsection{Expectation Values and Operators}
For a (one dimensional) particle in state $\psi$, the average value of its position $x$ is then
$$\langle x\rangle=\int_{\mathbb R}x\rho(x,t)\,\mathrm dx=\int_{\mathbb R}x|\bar\psi(x,t)|^2\,\mathrm dx$$
which can be interpreted as the average of repeated measurement on an ensemble of identically prepared systems.
Classically $p=m\dot{x}$ is the momentum.
In quantum mechanics, we define the average momentum to be
$$\langle p\rangle=m\frac{\mathrm d\langle x\rangle}{\mathrm dt}=m\frac{\mathrm d}{\mathrm dt}\int_{\mathbb R}x|\bar\psi(x,t)|^2\,\mathrm dx$$
We can calculate it and using the conservation rule we found earlier and assuming the boundary condition that $\psi$ decays fast enough as $x\to\pm\infty$,
\begin{align*}
    \langle p\rangle&=m\frac{\mathrm d}{\mathrm dt}\int_{\mathbb R}x|\bar\psi(x,t)|^2\,\mathrm dx\\
    &=m\int_{\mathbb R}x\frac{\partial(\bar\psi^*\bar\psi)}{\partial t}\,\mathrm dx\\
    &=\frac{i\hbar}{2}\int_{\mathbb R}x\frac{\partial}{\partial x}\left( \bar\psi^*\frac{\partial\bar\psi}{\partial x}-\bar\psi\frac{\partial\bar\psi^*}{\partial x} \right)\,\mathrm dx\\
    &=\frac{i\hbar}{2}\left[x\bar\psi^*\frac{\partial\bar\psi}{\partial x}-x\bar\psi\frac{\partial\bar\psi^*}{\partial x}\right]_{-\infty}^\infty-\frac{i\hbar}{2}\int_{\mathbb R}\left( \bar\psi^*\frac{\partial\bar\psi}{\partial x}-\bar\psi\frac{\partial\bar\psi^*}{\partial x} \right)\,\mathrm dx\\
    &=-\frac{i\hbar}{2}\int_{\mathbb R}\left( \bar\psi^*\frac{\partial\bar\psi}{\partial x}-\bar\psi\frac{\partial\bar\psi^*}{\partial x} \right)\,\mathrm dx\\
    &=-i\hbar\int_{\mathbb R}\bar\psi^\ast\frac{\partial\bar\psi}{\partial x}\,\mathrm dx
\end{align*}
By a bad notation one can write
$$\langle p\rangle=\int_{\mathbb R}\psi^*\left( -i\hbar\frac{\partial}{\partial x} \right)\psi\,\mathrm dx$$
To make it even worse we can consider the functional $\hat{x}$ and $\hat{p}$ defined by $\hat{x}=x$ and $\hat{p}=-i\hbar\partial/\partial x$, which allows us to effectively confuse everybody by writing
$$\langle x\rangle=\int_{\mathbb R}\psi^*\hat{x}\psi\,\mathrm dx,\langle p\rangle=\int_{\mathbb R}\psi^*\hat{p}\psi\,\mathrm dx$$
In general dimensions, we can use the same idea with $\hat{\underline{x}}=\underline{x}$ and $\hat{\underline{p}}=-i\hbar\nabla$.
This is where we desperately try to justify these notations by considering them as (linear) operators in the Hilbert space $\mathcal H$, which works mathematically.
\footnote{However, the notation shall haunt you for a considerable proportion of your relationship with quantum mechanics.}
The kinetic energy operator is then
$$\hat{T}=\frac{\hat{p}^2}{2m}=-\frac{\hbar^2}{2m}\nabla^2$$
In fact, for any physical quantity $Q(x,p)$ , we are just gonna write $\hat{Q}=Q(\hat{x},\hat{p})$ and we can get
$$\langle Q(x,p)\rangle=\int_{\mathbb R}\psi^*\hat{Q}\psi\,\mathrm dx=\int_{\mathbb R}\psi^*Q\left( x,-i\hbar\frac{\partial}{\partial x} \right)\psi\,\mathrm dt$$
Also, one can check by direct calculation that
$$\frac{\mathrm d\langle p\rangle}{\mathrm dt}=\int_{\mathbb R}\psi^*\left( -\frac{\partial U}{\partial x} \right)\psi\,\mathrm dx=\langle -U_x\rangle$$
by using TDSE.
The Hamiltonian operator is $\hat{H}=\hat{T}+\hat{U}$, so
$$(\hat{H}\psi)(\underline{x},t)=-\frac{\hbar^2}{2m}\nabla^2\psi+U\psi$$
\subsection{Time-Independent Schr\"odinger Equation}
Note that we can rewrite the Schr\"odinger equation in the form
$$i\hbar\frac{\partial\psi}{\partial t}=\hat{H}\psi$$
SO if we seperate the variables $\psi(\underline{x},t)=X(\underline{x})T(t)$, then by some rearrangement
$$i\hbar T^{-1}T^\prime=(\hat{H}X)/X$$
The left hand side depends only on $t$ while the right hand side on $\underline{x}$, requiring them to be equal is just saying they are actually constants.
Denote this constant by $E$, then we get
$$\begin{cases}
    i\hbar T^{-1}T^\prime=E\\
    \hat{H}X=EX
\end{cases}$$
The first equation is easy to solve and gives $T(t)=e^{-iEt/\hbar}$.
The second equation is called the Time-Independent Schr\"odinger Equation (TISE).
Its solution $X$ is then interpreted as a physical state with energy $E$.
Note that $E$ must be real by our expression of $T$ and the condition that $T$ does not explode as $t\to\pm\infty$.
Also, TISE is essentially the eigenvalue problem of $\hat{H}$ which makes sense as it is a linear operator on the vector space of states and $0$.
So we have obtained the set of solutions $\psi=Xe^{-iEt/\hbar}$ where $X$ is an eigenfunction of $\hat{H}$ with eigenvalue $E$.
This set of solutions is called the stationary states.
\subsection{Stationary States}
For a stationary state, assuming $\psi$ is normalised, then
$$\rho(\underline{x},t)=|\psi(\underline{x},t)|^2=|X(\underline{x})|^2|e^{-iEt/\hbar}|^2=|X(\underline{x})|^2$$
So the stationary states are some particular solutions of TDSE whose induced probability distributions in space do not depend on time.
Of course, superpositions of this family of stationary states is also an allowed state (or zero).
What's more, if $\psi_1=X_1e^{-iE_1t/\hbar},\psi_2=X_2e^{-iE_2t/\hbar}$ are stationary states and $\psi=a_1\psi_1+a_2\psi_2$ is a superposition, then (assuming $a_i$ and $X_i$ are real),
\begin{align*}
    |\psi|^2&=(a_1^*X_1^*e^{iE_1t/\hbar}+a_2^*X_2^*e^{iE_2t/\hbar})(a_1X_1e^{-iE_1t/\hbar}+a_2X_2e^{-iE_2t/\hbar})\\
    &=|a_1|^2|X_1|^2+|a_2|^2|X_2|^2+a_1a_2X_1X_2(e^{i(E_1-E_2)t/\hbar}+e^{i(E_2-E_1)t/\hbar})\\
    &=|a_1|^2|X_1|^2+|a_2|^2|X_2|^2+2a_1a_2X_1X_2\cos\left( \frac{(E_1-E_2)t}{\hbar} \right)
\end{align*}
So $\psi$ is not a stationary state if $E_1\neq E_2$ as $|\psi|^2$ depends on time.
In fact, the stationary state is a basis of the vector space of states (and zero), that is each state $\psi$ can be expressed in the form
$$\psi(\underline{x},t)=\sum_{n=1}^Na_nX_n(x)e^{-iE_nt/\hbar}$$
where $X_n,E_n$ are stuff you expect them to be.
We then interpret $|a_n|^2$ to be the probability for the energy to be $E_n$.
\begin{remark}
    If we have a discrete and normalisable basis $X_n$ of the Hamiltonian $\hat{H}$.
    Then we might write something of the form
    $$\psi=\sum_{n=1}^\infty a_nX_ne^{-iE_nt/\hbar}$$
    which is a solution to TDSE if we can show that it converges nice enough.
    But this will require
    $$\lim_{R\to\infty}\int_{|\underline{x}|>R}|X_n|^2\,\mathrm dx=0$$
    So the particle cannot be too far from the origin.
    We call this a bounded state.\\
    How about a continuous basis (i.e. the basis is indexed by $(X_\alpha)_{\alpha\in I}$ where $I$ is an interval)?
    Then we might write
    $$\psi=\int_{\alpha\in I}A(\alpha)X_\alpha(\underline{x})e^{-iE_\alpha t/\hbar}\,\mathrm d\alpha$$
    So we interpret $|A(\alpha)|^2\,\mathrm d\alpha$ as the probability for the state to have energy $E_\alpha$.
    But the same limit condition does not have to hold, since even if the state themselves are not normalisable, we can choose an $A$ that decays fast enough to make the eventual superposition normalisable.
    This is called a scattering state.
\end{remark}