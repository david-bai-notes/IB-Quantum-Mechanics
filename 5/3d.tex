\section{Schr\"odinger's Equation in Three Dimensions}
\subsection{TISE in 3D for Spherically Symmetric Potentials}
The TISE in 3D is in the form $\hat{H}X=EX$ where
$$\hat{H}=-\frac{\hbar^2}{2m}\nabla^2X+UX$$
In spherical coordiates $x=r\cos\phi\cos\theta,y=r\sin\phi\sin\theta,z=r\cos\theta$ where $r\in[0,\infty],\theta\in[0,\pi],\phi\in[0,2\pi)$, the Laplacian becomes
$$\nabla^2X=\frac{1}{r}\frac{\partial^2(rX)}{\partial r^2}+\frac{1}{r^2\sin^2\theta}\left( \sin\theta\frac{\partial}{\partial\theta}\left( \sin\theta\frac{\partial X}{\partial\theta} \right)+\frac{\partial^2X}{\partial\phi^2} \right)$$
We consider the case where $U=U(r,\theta,\phi)=U(r)$ and we are only looking for spherically symmetric solutions $X(r,\theta,\phi)=X(r)$.
Then the Hamiltonian becomes
$$\hat{H}X=-\frac{\hbar^2}{2m}\frac{1}{r}\frac{\mathrm d^2(rX)}{\mathrm dr^2}+UX=-\frac{\hbar^2}{2m}\left( \frac{\mathrm d^2X}{\mathrm dr^2}+\frac{2}{r}\frac{\mathrm dX}{\mathrm dr} \right)+UX$$
We have the normalisation condition
$$\int_{\mathbb R^3}|\psi^2|\,\mathrm dV<\infty\implies \int_0^\infty|X(r)|^2r^2\,\mathrm dr<\infty$$
which means $X(r)\to 0$ sufficiently fast as $r\to\infty$.\\
Consider $\sigma(r)=rX(r)$, then TISE becomes $\hat{H}\sigma=E\sigma$ on the half-plane $r\ge 0$.
Now the plan is to solve this on the whole line with $U(-r)=U(r)$.
As $X(0)$ are defined, $\sigma(0)=0$.
We only need to look for odd solutions $\sigma(-r)=-\sigma(r)$ since:
\begin{lemma}
    There is no nontrivial even solution to the problem.
\end{lemma}
\begin{proof}
    Suppose there is, say it is $\sigma_+$, then necessarily it is $C^2$ (for the equation to make sense) and hence $\sigma_+^\prime(0)=\sigma_+^{\prime\prime}(0)=0$.
    Then we have a corresponding odd solution
    $$\sigma_-(r)=\begin{cases}
        \sigma_+(r)\text{, for $r\ge 0$}\\
        -\sigma_+(r)\text{, for $r<0$}
    \end{cases}$$
    But then $\sigma(r)=\sigma_+(r)-\sigma_-(r)$ is a solution to the equation by linearity and vanishes for any $r>0$, which should not happen.
    \footnote{Well.}
\end{proof}
\begin{example}
    Consider the symmetric potential well
    $$U(r)=\begin{cases}
        0\text{, for $r\le a$}\\
        U_0\text{, for $r>a$}
    \end{cases}$$
    where $a,U_0>0$.
    For $0<E<U_0$, again let $k=\sqrt{2mE/\hbar^2}$ and $\bar{k}=\sqrt{2m(U_0-E)/\hbar^2}$, so
    $$\sigma(r)=\begin{cases}
        A\sin(kr)\text{, for $r\le a$}\\
        Be^{-\bar{k}r}\text{, for $r<a$}
    \end{cases}$$
    The boundary conditions gives $-k\cot(ka)=\bar{k}$ and we already know that $k^2+\bar{k}^2=2mU_0/\hbar^2$.
    Again define $\xi=ka,\eta=\bar{k}a$ and $r_0=a\sqrt{2mU_0/\hbar^2}$, so we have
    $$\begin{cases}
        -\xi\cot(\xi)=\eta\\
        \xi^2+\eta^2=r_0^2
    \end{cases}$$
    Now if $r_0<\pi/2$ (that is $U_0<\pi^2\hbar^2/(8ma^2)$), then there is no bounded solution.
    But unlike the one-dimensional case, we can always find at least one bounded state.
\end{example}
\subsection{The Radial Wavefunction of Hydrogen Atom}
The hydrogen atom consists of a proton $p^+$ and an electron $e^-$.
The Coulomb force is
$$F=-\frac{e^2}{4\pi\epsilon_0r^2}=-\frac{\partial U}{\partial r}\implies U(r)=-\frac{e^2}{4\pi\epsilon_0r}$$
So the bound states must have $E<0$, and $e^-$ is at rest $E=0$ as $r\to\infty$.
We want to look for stationary states of $e^-$ with spherically symmetric wavefunction $X=X(r)$ to
$$-\frac{\hbar^2}{2m}\left( \frac{\mathrm d^2X}{\mathrm dr^2}+\frac{2}{r}\frac{\mathrm dX}{\mathrm dr} \right)-\frac{e^2}{4\pi\epsilon_0r}X=EX$$
Then we need $E<0$ for bound states.
Define
$$\nu=\sqrt{-\frac{2mE}{\hbar^2}},\beta=\frac{e^2m}{2\pi\epsilon_0\hbar^2}$$
Then the equation becomes
$$\frac{\mathrm d^2X}{\mathrm dr^2}+\frac{2}{r}\frac{\mathrm dX}{\mathrm dr}+\left( \frac{\beta}{r}-\nu^2 \right)X=0$$
As $r\to\infty$, if we assume $X,X^\prime$ do not explode near $\infty$, then the equation becomes $X^{\prime\prime}-\nu^2X=0$, so $X(r)\sim e^{-\nu r}$ as $r\to\infty$.
This inspires us to try $X(r)=f(r)e^{-\nu r}$, which transforms the equation into
$$\frac{\mathrm d^2f}{\mathrm dr^2}+\frac{2}{r}(1-\nu r)\frac{\mathrm df}{\mathrm dr}+\frac{1}{r}(\beta-2\nu)f=0$$
which has a regular singular point at $r=0$, so we shall try the series solution $f(r)=r^c\sum_n a_nr^n$.
Plugging in,
$$0=\sum_{r=0}^\infty a_n(c+n)(c+n-1)r^{c+n-2}+\frac{2}{r}(1-\nu r)a_n(c+n)r^{c+n-1}+(\beta-2\nu)r^{c+n-1}$$
By looking at the coefficient of $r^{c-2}$, we have $0=a_0c(c+1)r^{c-2}$, so $c=-1$ or $c=0$.
The former is discarded as it has a singularity at $0$.
So $c=0$, hence $f(r)=\sum_na_nr^n$ and
$$0=\sum_{n=2}^\infty (a_nn(n+1)+a_{n-1}(\beta-2\nu))r^{n-2}=0\implies a_n=\frac{2\nu n-\beta}{n(n+1)}a_{n-1}$$
So all the coefficients are determined from $a_0$.
\begin{lemma}
    If the series above does not terminate, then the function $X(r)=f(r)e^{-\nu r}$ is not normalisable.
\end{lemma}
\begin{proof}
    If the series does not terminate then asymptotically $a_n\sim C(2\nu)^n/n!$ for some nonzero constant $C$ as $n\to\infty$.
    Therefore $f(r)\sim e^{2\nu r}$, hence $X(r)\sim e^{\nu r}$ as $r\to\infty$ which means that it is not normalisale.
\end{proof}
So necessarily there is some $N$ such that $2\nu N-\beta=0$, so $\nu=\beta/(2N)$.
We can work out the energy eigenfunction from that:
$$\nu^2=-\frac{2mE}{\hbar^2},\beta=\frac{e^2m}{2\pi\epsilon_0\hbar^2}\implies E=E_N=-\frac{e^4m}{32\pi^2\epsilon_0^2\hbar^2}\frac{1}{N^2}$$
which is exactly the energy levels deduced by Bohr's radius.
Would this mean that Bohr's model is fully correct?
Not really, as there can be solutions that are not spherically symmetric.\\
Let's find the eigenfunctions now.
Plugging the relation between $N,\nu$ and $\beta$ we have
$$\frac{a_n}{a_{n-1}}=-2\nu\frac{N-n}{n(n+1)}$$
$N=1$ gives a constant $f=f_1$, so $X_1(r)\propto e^{-\nu r}$.
$N=2$ gives $f_2(r)\propto 1-\nu r$, so $X_2(r)\propto (1-\nu r)e^{-\nu r}$.
$N=3$ gives $f_3(r)\propto 1-2\nu r+2\nu^2r^2/3$, so $X_3(r)=(1-2\nu r+2\nu^2r^2/3)e^{-\nu r}$.
And in general $f_N(r)=L_N(\nu r)$ where $L_N$ is the Legendre polynomial of order $N-1$ and hence $X_N(r)=L_N(\nu r)e^{-\nu r}$.
It is easy enough to normalise this for small $N$.
For the ground state $N=1$, the normalised wavefunction is
$$X_1(r)=\frac{\nu^{3/2}}{\sqrt{\pi}}e^{-2\nu r}=\frac{1}{\sqrt\pi}\left( \frac{e^2m}{4\pi \epsilon_0\hbar^2} \right)^{3/2}e^{-2\nu r}$$
With a little calculation we have $\langle r\rangle_{X_1}=3a_0/2$ where $a_0=2/\beta$ is the Bohr radius.
\subsection{Angular Momentum in Quantum Mechanics}
In classical dynamics, we defined the angular momentum $\underline{L}=\underline{x}\times\underline{p}$ which is a conserved quantity subject to spherically symmetrical potential.
Whereas in quantum mechanics, we have an analogous notion.
\begin{definition}
    The angular momentum operator is defined by $\underline{\hat{L}}=\underline{\hat{x}}\times\underline{\hat{p}}=-i\hbar\underline{x}\times\nabla$.
\end{definition}
Consequently components of $\underline{\hat{L}}$ are Hamiltonian and $[\hat{L}_i,\hat{L}_j]=i\hbar\epsilon_{ijk}\hat{L}_k\neq 0$ for $i\neq j$.
Therefore components of $\underline{\hat{L}}$ cannot be simultaneously measured.
\begin{definition}
    The total angular momentum operator is defined as $\hat{L}^2=|\hat{L}|^2=\hat{L}_1^2+\hat{L}_2^2+\hat{L}_3^2$.
\end{definition}
Then $[\hat{L}^2,\hat{L}_i]=0$ for all $i$ and $[\hat{H},\hat{L}^2]=0$ if $U$ is spherically symmetric.
To see the latter property, just observe that $[\hat{L}_i,\hat{x}_j]=i\hbar\epsilon_{ijk}\hat{x}_k$ and $[\hat{L}_i,\hat{p}_j]=i\hbar\epsilon_{ijk}\hat{p}_k$ and write $\hat{H}=\hat{p}^2/2m+\hat{U}$.
Consequently, fixing any $i$, $\{\hat{H},\hat{L}^2,\hat{L}_i\}$ is a set of $3$ mutually commuting operators.
Choose $i=3$, then $\hat{L}_3=\hat{L}_z$ is the $z$ component of angular momentum.
We can find simultaneous eigenstates of all three operators $\{\hat{H},\hat{L}^2,\hat{L}_3\}$.
The corresponding eigenvalues are observables.
Also, this set if maximal in the sense that there does not exists another independent operator other than $\hat{I}$ which commutes with all three.\\
What are the eigenfunctions?
We first try to find joint eigenfunctions of $\hat{L}^2,\hat{L}_3$.
In spherical polar coordinate, we can expand
$$\hat{L}^2=-\frac{\hbar^2}{\sin^2\theta}\left( \sin\theta\frac{\partial}{\partial\theta}\left( \sin\theta\frac{\partial}{\partial\theta} \right)+\frac{\partial^2}{\partial\phi^2} \right),\hat{L}_3=-i\hbar\frac{\partial}{\partial\phi}$$
If $Y=Y(\theta,\phi)$ is an eigenfunction of $\hat{L}_3$, then $\hat{L}_3Y=\hbar mY$ for some $m$.
Consequently $-iY_\phi=mY$.
Seperation of variables $Y(\theta,\phi)=y(\theta)x(\phi)$ gievs $x_\phi=imx$, so $x(\phi)=e^{im\phi}$.
But $x$ has to have period $2\pi$, so $m\in\mathbb Z$.
Now $\hat{L}^2(y(\theta)e^{im\phi})=\lambda y(\theta)e^{im\phi}$ gives
$$\frac{1}{\sin\theta}\frac{\partial}{\partial\theta}\left( \sin\theta\frac{\partial y}{\partial\theta} \right)-\frac{m^2}{\sin^2\theta}y(\theta)=-\frac{\lambda}{\hbar^2}y(\theta)$$
Which is Legendre equation.
By our study in Methods, we have
$$y_l(\theta)=P_{l,m}(\cos\theta)=(\sin\theta)^{|m|}\left.\frac{\mathrm d^{|m|}}{\mathrm dx^{|m|}}P_l(x)\right|_{x=\cos\theta}$$
where $P_l$ are the ordinary Legendre polynomials and $\lambda=l(l+1)\hbar^2$ for $-l\le m\le l$.
So the simultaneous eigenfunctions of $\hat{L}^2$ and $\hat{L}_3$ are labelled by $l=0,1,2,\ldots,m\in\{-l,\ldots,l\}$ and take the form
$$y_{l,m}(\theta,\phi)=P_{l,m}(\cos\theta)e^{im\phi}$$
with eigenvalues $m\hbar$ and $l(l+1)\hbar^2$ respectively.
Physically, we interpret $l$ as the total angular momentum numbers and $m$ the azimuthal quantum numbers.
Note that the constraint $m\in[-l,l]$ corresponds to the classical mechanical result of $-|L|\le L_3\le |L|$.\\
We can compute (whyyyyyyyy) some of these eigenfunctions (known as spherical harmonics):
$$Y_{0,0}=\frac{1}{\sqrt{4\pi}},Y_{1,0}=\sqrt{\frac{3}{4\pi}}\cos\theta,Y_{1,\pm 1}(\theta,\phi)=\mp\sqrt{\frac{3}{8\pi}}\sin\theta e^{\pm i\phi}$$
$$Y_{2,0}(\theta,\phi)=\sqrt{\frac{5}{16\pi}}(3\cos^2\theta-1),Y_{2,\pm 1}(\theta,\phi)=\mp\sqrt{\frac{15}{8\pi}}\sin\theta\cos\theta e^{\pm i\phi}$$
$$Y_{2,\pm 2}(\theta,\phi)=\sqrt{\frac{15}{32\pi}}\sin^2\theta e^{\pm 2i\phi}$$
\subsection{Full Wavefunction of the Hydrogen Atom}
Recall that in spherical polar,
$$\nabla^2X=\frac{1}{r}\frac{\partial^2(rX)}{\partial r^2}+\frac{1}{r^2\sin^2\theta}\left( \sin\theta\frac{\partial}{\partial\theta}\left( \sin\theta\frac{\partial X}{\partial\theta} \right)+\frac{\partial^2X}{\partial\phi^2} \right)$$
and
$$\hat{L}^2X=\frac{\hbar^2}{\sin^2\theta}\left( \sin\theta\frac{\partial}{\partial\theta}\left( \sin\theta\frac{\partial X}{\partial\theta} \right) +\frac{\partial^2}{\partial\phi^2}\right)$$
Therefore
$$-\hbar^2\nabla^2X=-\frac{\hbar^2}{r}\frac{\partial^2(rX)}{\partial r^2}+\frac{\hat{L}^2X}{r^2}$$
which means that we can rewrite the Hamiltonian as
$$\hat{H}X=-\frac{\hbar^2}{2m}\left( \frac{\partial^2X}{\partial r^2}+\frac{2}{r}\frac{\partial X}{\partial r} \right)+\frac{\hat{L}^2X}{2mr^2}-\frac{e^2}{4\pi\epsilon_0r}X$$
As $\{\hat{H},\hat{L}^2,\hat{L}_3\}$ is a set of commuting operators, we look for solutions to $\hat{H}X=EX$ of the form $X(r,\theta,\phi)=g(r)Y_{l,m}(\theta,\phi)$, which (after a lot of calculations) yields
$$\frac{\mathrm d^2g}{\mathrm dr^2}+\frac{2}{r}\frac{\mathrm dg}{\mathrm dr}-\frac{l(l+1)}{r^2}g+\frac{\beta}{r}g-\nu^2g=0,\nu=\sqrt{-\frac{2mE}{\hbar^2}},\beta=\frac{e^2m}{2\pi\epsilon_0\hbar^2}$$
The rest is similar to what we did for the spherically symmetrical case.
We seek solutions of the form $g(r)=f(r)e^{-\nu r}$ due to its limiting behavour, which gives
$$\frac{\mathrm d^2f}{\mathrm dr^2}+\frac{2}{r}(1-\nu r)\frac{\mathrm df}{\mathrm dr}-\frac{l(l+1)}{r^2}f+\frac{1}{r}(\beta-2\nu)f=0$$
The same procedure of finding series solution applies.
Since $0$ is a regular singular point, we try solutions of the form $f(r)=r^\sigma\sum_na_nr^n$.
By routine work of substituting we get the indicial equation $\sigma(\sigma-1)+2\sigma-l(l+1)=0$, hence $\sigma=l$ or $\sigma=-l-1$.
The latter case has to be discarded as we don't want singularity at $0$.
Therefore $f(r)=r^l\sum_na_nr^n$.
Again routine substitution work yields the recurrence
$$a_n=\frac{2\nu(n+l)-\beta}{n(n+2l+1)}a_{n-1}$$
which determines the series completely up to a constant.
The same trick as before applies:
If the series does not terminate, then by estimating the order of growth of $(a_n)$ we find that the $g$ that we want will not be normalisable.
Therefore the series must terminate.
This means that (for nonzero $a_0$) there exists some $n_{\max{}}$ such that $a_{n_{\max{}}}=0$ but $a_{n_{\max{}}-1}\neq0$ which has to satisfy
$$2\nu(n_{\max{}}+l)-\beta=0\implies \nu=\frac{\beta}{2N},N=n_{\max{}}+l$$
by plugging in the respective definition, the energy levels are
$$E=E_N=-\frac{e^4m}{32\pi^2\epsilon_0^2\hbar^2}\frac{1}{N^2}$$
So the spectrum of energy levels is identical to what we obtained in the special case where the eigenfunctions are spherically symmetric.
But in this more general setting, the eigenvalues can have some other (non-spherically-symmetric) eigenfunctions.
How many of them?
It's not hard to see:
For each $N$, there are exactly
$$D(N)=\sum_{l=0}^{N-1}\sum_{n=-l}^l1=\sum_{l=0}^{N-1}(2l+1)=N^2$$
many different eigenfunctions.
This is called the degeneracy of the energy value $E_N$.
So the full spectrum of the eigenfunctions of the Hydrogen atom is then
$$X_{N,l,m}(r,\theta,\phi)=\xi^lL_{N,l}(\xi)e^{-\xi}Y_{l,m}(\theta,\phi),\xi=\frac{\beta r}{2N}=\frac{e^2mr}{4N\pi\epsilon_0\hbar^2}$$
where $L_{N,l}$ (whose coefficients are defined by our terminating series earlier) are called the generalised Laguerre polynomials and $Y_{l,m}$ are the spherical harmonics.\\
$N=1,2,3,\ldots$ are called the principal quantum numbers, $l=0,\ldots,N-1$ are the total angular momentum quantum numbers and $m\in\{-l,\ldots,l\}$ are the azimuthal quantum numbers.
The Bohr model captured the case where $m=l\simeq N>>1$.
In this case $L_3==m\hbar\simeq N\hbar$ and $L=\sqrt{L^2}=\sqrt{l(l+1)\hbar^2}\simeq N\hbar$.\\
Note that due to the definition of the spherical harmonics, we have
$$\int_0^{2\pi}\int_{-1}^1\int_0^\infty |X_{N,l,m}(r,\theta,\phi)|^2r^2\,\mathrm dr\mathrm d(\cos\theta)\mathrm d\phi=\int_0^\infty |g(r)|^2r^2\,\mathrm dr$$
So we define $P(r)|g(r)|^2r^2$ as the radial probability.
Asymptotically,
$$P(r)=r^2|g(r)|^2\sim r^{2(l+1)}\exp\left( -\frac{\beta r}{2(l+1)} \right)$$
which, in the Bohr limit, just gives $r^{2N}\exp(-\beta r/N)$.
By differentiating, the stationary points of $P$ (i.e. the modes) are approximately at $r_{\rm peak}\simeq 2N^2/\beta=N^2r_1$ which is the same as what we obtained using Bohr's model.
\subsection{Towards Periodic Table}
If we have a nucleus with charge $+ze$ and there are $z$ electrons around it which are treated as independent particles (so ignore any interactions betweem electrons like electromagnetism and gravity), then the eigenfunction can be seperated as $X(\underline{x}_1,\ldots,\underline{x}_z)=X(\underline{x}_1)\cdots X(\underline{x}_z)$ where each $X(\underline{x}_i)$ is a rescaled solution for the Hydrogen atom.
Its energy level is then the sum of energies of all electrons $E=\sum_iE_i$.
which works for small $z$, but the interaction force between electrons is growing when $z$ is large -- which is why we need to develop more theory in next year's Principle of Quantum Mechanics.